\begin{enumerate}

    \item Introduction
    \begin{enumerate}
        \item Hook, summary of alpha-bimodality
        \item Summary of two-infall literature
        \item Summary of secular evolution models which claim alpha-bimodality
        \item Goal of the paper
    \end{enumerate}
    
    \item Methods
    \begin{enumerate}
        \item Description of GCE models
        \begin{enumerate}
            \item {\it Plot:} Two-infall parameters in one-zone models. The second infall timescale $\tau_2$ is the most important parameter for fitting the low-alpha sequence and the MDF.
        \end{enumerate}
        \item Motivate outflow prescription
        \item Discuss yields
        \begin{enumerate}
            \item {\it Plot:} Yield scaling in one-zone models. No yield set is fully compatible with the observed abundance history, so we investigate both the high yield and low yield cases in our multi-zone models.
        \end{enumerate}
        \item Describe data sources
    \end{enumerate}
    
    \item Multi-zone model results
    \begin{enumerate}
        \item Stellar abundance distributions
        \begin{enumerate}
            \item {\it Plot:} Stellar [O/Fe] distributions from multi-zone models with the low and high yield sets. Using the higher yields, the global [O/Fe] distribution is triple-peaked which does not fit the data.
            \item {\it Plot:} Abundance diagram from a high-yield one-zone model which illustrates that the turn-over in [O/Fe] evolution due to the second infall epoch produces an intermediate [O/Fe] peak in the stellar distribution.
        \end{enumerate}
        \item The radial abundance gradient
        \begin{enumerate}
            \item {\it Plot:} Radial abundance gradients predicted by multi-zone models with low and high yields.
            \item A variable second infall timescale
            \item Altering the outflow prescription
        \end{enumerate}
        \item Abundance histories
        \begin{enumerate}
            \item {\it Plot:} Comparison of stellar age-abundance diagrams from multi-zone models with low and high yields. The second infall epoch produces substantial dilution of the ISM metallicity, for which there is little observational evidence. Higher yields help the system recover from perturbations more rapidly and are in less disagreement with the data.
            \item {\it Plot:} Similar, but showing the mode of the stellar abundance distributions over time compared to the data. Due to the dilution from the second infall, the Solar neighborhood spends less time in equilibrium in the high-yield case, and doesn't yet reach equilibrium in the low-yield case.
            \item Can CGM pre-enrichment mitigate dilution and/or approach to equilibrium issues?
            \item Other parameters that can mitigate these issues
        \end{enumerate}
    \end{enumerate}
    
    \item Discussion
    \begin{enumerate}
        \item Other ways to fix the thin disk abundances
        \begin{enumerate}
            \item {\it Plot:} The DTD in one-zone models. A prompt DTD can reduce the size of the low-alpha loop, but it messes up the high-alpha sequence.
            \item {\it Plot:} Adjusting the thick-to-thin disk prescription in one-zone models. If the mass in the first infall is significantly more substantial, the [O/Fe] distribution better matches the data in the high yield case.
        \end{enumerate}
        \item Radial gas flows
        \item SF hiatus not driven by infall history
        \begin{enumerate}
            \item {\it Plot:} An SFE-driven hiatus in a one-zone model can produce alpha-bimodality without altering the gas inflow prescription.
        \end{enumerate}
    \end{enumerate}
    
    \item Conclusions
    \begin{enumerate}
        \item Summary of findings
        \begin{enumerate}
            \item Attempts to fix the problem with the [O/Fe] distribution in the thin disk end up creating issues with other observables, such as the thick/thin disk surface density ratio, the SN Ia DTD, or the age-metallicity relation.
            \item The equilibrium model, if accurate, places very strict limits on the two-infall model, moreso than for other models.
            \item A short ($\sim200$ Myr) star formation hiatus early in the history of the disk is sufficient to produce a bimodal distribution of [O/Fe] without altering the gas infall prescription. This is distinct from the star formation hiatus produced by the two-infall model.
        \end{enumerate}
    \end{enumerate}
\end{enumerate}

\subsection{Summary of Two-Infall Papers}

% To add: Carme Gallart (2019) three-infall-ish, Kobayashi metallicity limit for SNe Ia and sub-Chandra

% Summary of two-infall papers
\citet{matteucci_galactic_1989} modeled the galaxy with a two-component model, a one-zone model for the halo and concentric shell zones for the disk. They used a simple exponential gas infall (so not yet a two-infall model), where the infall timescale increased proportionally with radius. They studied cases with primordial and pre-enriched gas infall, finding that the former produces a more realistic metallicity distribution in the Solar neighborhood. Their inside-out disk evolution produced good agreement with observed abundance gradients.

\citet{chiappini_chemical_1997} presented the first two-infall model of the MW disk. They adopted an inside-out prescription for the formation of the thin disk, with the timescale of the second infall increasing linearly with radius. They assume the formation timescale of the thick disk is 1 Gyr, and the time of the second infall is 2 Gyr. They also assume a lower threshold on the gas density required for star formation, which produces a star formation hiatus at the end of the thick disk phase. They also noted the ``loop'' that appears in the \aFe--[Fe/H] relation during the thin disk phase. They mostly compare to Solar neighborhood data.

\citet{chiappini_abundance_2001} updated their model to fit the observed abundance, stellar, and gas gradients in the Milky Way disk. They find that a density threshold for star formation, making a gap in star formation between the halo and disk phases, is necessary to reproduce observational data. They argue that the details of halo formation also have a strong impact on the outer disk. No radial flows or outflows. I'm confused about their distinction between halo and thick disk, if any.

\citet{chiappini_oxygen_2003} compare model predictions of C, N, and O to Milky Way and other spiral galaxy data. They predict a very slow enrichment between the birth of the Sun and present day, which they claim is due to their adopted gas density threshold for star formation (the infall rate is low enough that the SFR oscillates from timestep to timestep).

\citet{matteucci_effect_2009} investigated different SN Ia delay time distributions using the \citep{chiappini_chemical_1997,chiappini_abundance_2001} model. They use a model for the full disk but compare only to Solar neighborhood data. They concluded that some prompt SN Ia enrichment is necessary to match observed abundance patterns.

\citet{spitoni_effects_2009} relaxed the instantaneous recycling approximation in two different ways: galactic fountains (i.e., SN ejecta are launched out of the disk and take some time to return) and metal cooling (i.e., newly-enriched hot gas takes time to cool before it can form stars). They found that fountains do not strongly affect the evolution of the $\alpha$-elements for short (few hundred Myr) timescales. The effect of a metal cooling timescale is similarly negligible if the yields are not dependent on metallicity.

\citet{grisoni_ambre_2017} argue for a ``parallel'' scenario, where the thick and thin disks form simultaneously but with very different infall timescales (0.1 and 7 Gyr, respectively). They compare their models with stellar abundances from the AMBRE sample of solar-neighborhood dwarfs. They argue that the parallel scenario can produce metal-rich high-alpha stars in situ, whereas in the two-infall scenario these stars would have to migrate from the inner disk.

\citet{spitoni_galactic_2019} compare a two-infall model for the Solar neighborhood to the parallel infall scenario of \citet{grisoni_ambre_2017}. They argue that the two-infall model reproduces the observed age distribution of the high-$\alpha$ and low-$\alpha$ sequences and the age--metallicity relation, but they argue that it requires a longer delay of $\sim4$ Gyr between the first and second infalls.

\citet{palla_chemical_2020} explored radial gas flows, variable star formation efficiency of the thin disk, and the onset time of the second infall (but not the infall timescales). They find that inside-out formation (i.e., longer infall timescales in the outer disk) is not sufficient to reproduce observed abundance gradients, and that radial gas flows, possibly in combination with a radially-varying star formation law, are key. Pre-enriched infall in the inner disk ($R<8\,{\rm kpc}$) at the level of -0.5 dex also improves agreement with data.

% Three-infall
Motivated by constraints on the SFH by {\it Gaia}, \citet{spitoni_beyond_2023} proposed the three-infall model, an extension with a third phase of exponential gas infall in the last few Gyr. They argue that the Gaia DR3 data show a population of young stars with sub-Solar abundances which is predicted by the most recent infall.

\citet{palla_mapping_2024} compare model predictions to abundances and ages of open clusters from the {\it Gaia}-ESO survey. They find that the two-infall model over-predicts the metallicity of the youngest clusters, a problem which is fixed by the three-infall model. A version of the three-infall model with milder dilution (mild pre-enrichment of the gas with a smaller amount of infall) compared to \citet{spitoni_beyond_2023} can simultaneously match the metalliticy of old, intermediate, and young clusters across the disk.

\citet{spitoni_remind_2024} argue that the high- and low-$\alpha$ sequences being separated in [$\alpha$/Fe] but quite similar in [$\alpha$/H] implies a hiatus in star formation, which is produced by the two-infall model of \citet{spitoni_galactic_2019}. They look at birth radius estimates and find evidence for two modes in the low-$\alpha$ sequence. They use pre-enriched gas infall at the level of $\rm{[Fe/H]}=-0.8$.
