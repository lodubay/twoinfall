\begin{enumerate}

    \item Introduction
    \begin{enumerate}
        \item Hook, summary of alpha-bimodality
        \item Summary of two-infall literature
        \item Summary of secular evolution models which claim alpha-bimodality
        \item Goal of the paper
    \end{enumerate}
    
    \item Methods
    \begin{enumerate}
        \item Description of GCE models
        \begin{enumerate}
            \item {\it Plot:} Two-infall parameters in one-zone models. The second infall timescale $\tau_2$ is the most important parameter for fitting the low-alpha sequence and the MDF.
        \end{enumerate}
        \item Motivate outflow prescription
        \item Discuss yields
        \begin{enumerate}
            \item {\it Plot:} Yield scaling in one-zone models. No yield set is fully compatible with the observed abundance history, so we investigate both the high yield and low yield cases in our multi-zone models.
        \end{enumerate}
        \item Describe data sources
    \end{enumerate}
    
    \item Multi-zone model results
    \begin{enumerate}
        \item Stellar abundance distributions
        \begin{enumerate}
            \item {\it Plot:} Stellar [O/Fe] distributions from multi-zone models with the low and high yield sets. Using the higher yields, the global [O/Fe] distribution is triple-peaked which does not fit the data.
            \item {\it Plot:} Abundance diagram from a high-yield one-zone model which illustrates that the turn-over in [O/Fe] evolution due to the second infall epoch produces an intermediate [O/Fe] peak in the stellar distribution.
        \end{enumerate}
        \item The radial abundance gradient
        \begin{enumerate}
            \item {\it Plot:} Radial abundance gradients predicted by multi-zone models with low and high yields.
            \item A variable second infall timescale
            \item Altering the outflow prescription
            \item Radial gas flows
        \end{enumerate}
        \item Abundance histories
        \begin{enumerate}
            \item {\it Plot:} Comparison of stellar age-abundance diagrams from multi-zone models with low and high yields. The second infall epoch produces substantial dilution of the ISM metallicity, for which there is little observational evidence. Higher yields help the system recover from perturbations more rapidly and are in less disagreement with the data.
            \item {\it Plot:} Similar, but showing the mode of the stellar abundance distributions over time compared to the data. Due to the dilution from the second infall, the Solar neighborhood spends less time in equilibrium in the high-yield case, and doesn't yet reach equilibrium in the low-yield case.
            \item Can CGM pre-enrichment mitigate dilution and/or approach to equilibrium issues?
            \item Other parameters that can mitigate these issues
        \end{enumerate}
    \end{enumerate}
    
    \item Discussion
    \begin{enumerate}
        \item Other ways to fix the thin disk abundances
        \begin{enumerate}
            \item {\it Plot:} The DTD in one-zone models. A prompt DTD can reduce the size of the low-alpha loop, but it messes up the high-alpha sequence.
            \item {\it Plot:} Adjusting the thick-to-thin disk prescription in one-zone models. If the mass in the first infall is significantly more substantial, the [O/Fe] distribution better matches the data in the high yield case.
        \end{enumerate}
        \item SF hiatus not driven by infall history
        \begin{enumerate}
            \item {\it Plot:} An SFE-driven hiatus in a one-zone model can produce alpha-bimodality without altering the gas inflow prescription.
        \end{enumerate}
    \end{enumerate}
    
    \item Conclusions
    \begin{enumerate}
        \item Summary of findings
        \begin{enumerate}
            \item Attempts to fix the problem with the [O/Fe] distribution in the thin disk end up creating issues with other observables, such as the thick/thin disk surface density ratio, the SN Ia DTD, or the age-metallicity relation.
            \item The equilibrium model, if accurate, places very strict limits on the two-infall model, moreso than for other models.
            \item A short ($\sim200$ Myr) star formation hiatus early in the history of the disk is sufficient to produce a bimodal distribution of [O/Fe] without altering the gas infall prescription. This is distinct from the star formation hiatus produced by the two-infall model.
        \end{enumerate}
    \end{enumerate}
\end{enumerate}