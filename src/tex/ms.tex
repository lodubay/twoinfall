% For TeXworks
\makeatletter
\declare@file@substitution{revtex4-1.cls}{revtex4-2.cls}
\makeatother

% Define document class
\documentclass[trackchanges,twocolumn,twocolappendix,linenumbers]{aastex631}
% \documentclass[linenumbers]{aastex631}

% \usepackage{showyourwork}
\usepackage{amsmath}
\usepackage{amssymb}
\usepackage{graphicx}
\usepackage{xcolor}
\let\tablenum\relax
\usepackage{siunitx}
\usepackage{xspace}
% Dropped capital
\usepackage{lettrine}

% ***Fancy*** dropped capital
\input Zallman.fd
\renewcommand{\LettrineFontHook}{\usefont{U}{Zallman}{xl}{n}}
\setcounter{DefaultLines}{3} % number of lines dropped capital should take up

% user-defined commands
\newcommand{\GitHubURL}{https://github.com/lodubay/twoinfall}
\newcommand{\vice}{{\tt VICE}\xspace}
\newcommand{\todo}[1]{{\color{red}#1}}
\newcommand{\osuaffil}{Department of Astronomy, The Ohio State University, 140 W. 18th Ave, Columbus OH 43210, USA}
\newcommand{\ccappaffil}{Center for Cosmology and AstroParticle Physics, The Ohio State University, 191 W. Woodruff Ave., Columbus OH 43210, USA}
\newcommand{\aFe}{[$\alpha$/Fe]\xspace}
\newcommand{\mathXH}{{\rm [X/H]}}
\newcommand{\mathOH}{{\rm [O/H]}}
\newcommand{\mathFeH}{{\rm [Fe/H]}}
\newcommand{\mathOFe}{{\rm [O/Fe]}}
\newcommand{\mathaFe}{[\alpha/{\rm Fe}]}
\newcommand{\yZ}[1]{$y/Z_\odot=#1$}
\newcommand{\kpc}{\,{\rm kpc}}
\newcommand{\Myr}{\,{\rm Myr}}
\newcommand{\Gyr}{\,{\rm Gyr}}
\newcommand{\dex}{\,{\rm dex}}
\newcommand{\Msun}{\,{\rm M}_\odot}
\newcommand{\kms}{\,{\rm km}\,{\rm s}^{-1}}
\newcommand{\onecolumn}{0.47\textwidth}

% Citation aliases
\defcitealias{johnson_stellar_2021}{J21}
\defcitealias{leung_variational_2023}{L23}
\defcitealias{dubay_galactic_2024}{D24}

\shorttitle{Two-Infall Challenges}
\shortauthors{Dubay et al.}
% \linespread{1.8}

\submitjournal{The Astrophysical Journal}

\begin{document}

% Title
\title{Challenges to the Two-Infall Scenario by Large Stellar Age Catalogs}

% Author list
\author[0000-0003-3781-0747]{Liam O.\ Dubay}
\affiliation{\osuaffil}
\affiliation{\ccappaffil}
\author[0000-0001-7258-1834]{Jennifer A.\ Johnson}
\affiliation{\osuaffil}
\affiliation{\ccappaffil}
\author[0000-0002-6534-8783]{James W.\ Johnson}
\affiliation{Observatories of the Carnegie Institution for Science, 813 Santa Barbara St., Pasadena CA 91101, USA}
\author[0000-0002-2854-5796]{John D.\ Roberts}
\affiliation{\osuaffil}
\affiliation{\ccappaffil}

\correspondingauthor{Liam O.\ Dubay}
\email{dubay.11@osu.edu}

\begin{abstract}
    Stars in the Milky Way disk exhibit a clear separation into two chemically distinct populations based on their [$\alpha$/Fe] ratios. This $\alpha$-bimodality is not a universal feature of simulated disk galaxies and may point to a unique evolutionary history. A popular \added{and well-studied} explanation is the two-infall scenario, which postulates that two periods of substantial accretion rates dominate the assembly history of the Galaxy. Thanks to recent advances in stellar age measurements, we can now compare this model to more direct measurements of the Galaxy's evolutionary timescales across the disk. We run multi-zone galactic chemical evolution models with a two-infall-driven star formation history and compare the results against abundance patterns from APOGEE DR17, supplemented with stellar ages estimated through multiple methods. Although the two-infall scenario offers a natural explanation for the $\alpha$-bimodality, it struggles to explain several features of the age--abundance structure in the disk. First, our models generically require a massive and long-lasting dilution event, but the data show that stellar metallicity is remarkably constant across much of the lifetime of the disk. This apparent age-independence places considerable restrictions upon the two-infall parameter space. Second, most local metal-rich stars in APOGEE have intermediate ages, yet our models predict these stars should either be very old or very young. Some of these issues can be mitigated, but not completely resolved, by pre-enriching the accreted gas to low metallicity. These restrictions also place limits on the role of merger events in shaping the chemical evolution of the thin disk.
\end{abstract}

\section{Introduction}
\label{sec:introduction}

\lettrine{G}{alactic chemical evolution} (GCE) \added{studies aim} to explain the metal abundance patterns observed in the Milky Way (MW) by modeling the star formation history and evolution of the Galaxy. A long-standing paradigm of GCE is that the metallicity of the interstellar medium (ISM) increases over time due to supernova enrichment from successive generations of stars \citep[e.g.,][]{tinsley_stellar_1979,matteucci_relative_1986}. In this view, one feature of the MW disk that is difficult to explain is the so-called ``$\alpha$-bimodality'': the presence of two populations of stars at similar metallicity but separated by their \aFe ratio \citep[e.g.,][]{bensby_exploring_2014}. The high-$\alpha$ sequence consists of old stars \citep[$\gtrsim9$ Gyr; e.g.,][]{pinsonneault_apokasc-3_2025} with super-Solar \aFe \added{that are} associated with the kinematic thick disk \citep[e.g.,][]{fuhrmann_nearby_1998}, while the low-$\alpha$ sequence is younger, has approximately Solar \aFe, and is associated with the kinematic thin disk. The $\alpha$-bimodality is present across the Galactic disk, but the relative strength of the high- and low-$\alpha$ sequences varies by location \citep{hayden_chemical_2015}. \added{An $\alpha$-bimodality is not a universal feature in simulated MW-mass galaxies \citep[e.g.,][]{mackereth_origin_2018,parul_effect_2025}, and may not exist in M31 (\citealt{nidever_prevalence_2024}; but see also \citealt{kobayashi_fe_2023}), suggesting that it could point to a unique evolutionary history for the MW.}

% An explanation for the MW $\alpha$-bimodality has yet to be broadly accepted in the GCE literature. An $\alpha$-bimodality is not a universal feature in simulated MW-mass galaxies \citep[e.g.,][]{mackereth_origin_2018,parul_effect_2025}, and may not exist in M31 (\citealt{nidever_prevalence_2024}; but see also \citealt{kobayashi_fe_2023}). Its presence and characteristics in our Galaxy could potentially have arisen due to a unique evolutionary history. GCE models that attempt to solve this problem generally fall into two camps. Some explain the $\alpha$-bimodality as a result of secular processes, such as the radial migration of stars or the inside-out growth of the disk \citep[e.g.,][]{schonrich_chemical_2009,kubryk_evolution_2015,sharma_chemical_2021,chen_chemical_2023,prantzos_origin_2023}. Others argue for a bursty star formation history driven by multiple accretion events \citep[e.g.,][]{chiappini_chemical_1997,mackereth_origin_2018,spitoni_beyond_2023}.

\added{The two-infall scenario} was first proposed by \citet{chiappini_chemical_1997} to explain the origin of the \added{$\alpha$-bimodality in the MW. The basic premise is that the rate of gas accretion onto the disk occurs in two bursts}. The relatively low infall rate between the two bursts causes star formation to slow down, allowing the gas abundance to evolve between the high- and low-$\alpha$ sequences while producing few stars. 
The infall timescale for the low-$\alpha$ disk can vary with radius to produce inside-out disk growth and a radial metallicity gradient \citep{romano_mass_2000}.
The initial model of \citet{chiappini_chemical_1997} successfully reproduced the \added{Solar neighborhood abundance data available at the time}.

Subsequent studies refined the two-infall scenario to reproduce abundance data across the disk \citep[e.g.,][]{chiappini_abundance_2001,chiappini_oxygen_2003}, \added{as well as the local surface densities of stars and gas and the local star formation and supernova rates \citep[e.g.,][]{romano_mass_2000,spitoni_galactic_2020,spitoni_remind_2024}}. Others have explored the SN Ia delay-time distribution \citep{matteucci_effect_2009,palicio_analytic_2023}, galactic fountains \citep{spitoni_effects_2009}, radial gas flows \citep{spitoni_effects_2011,palla_chemical_2020}, variations in the star formation efficiency \citep{spitoni_effects_2011,palla_chemical_2020}, radial stellar migration \citep{spitoni_effect_2015,palla_mgfe_2022}, azimuthal abundance variations \citep{spitoni_2d_2019}, and pre-enriched gas infall \citep{palla_chemical_2020,spitoni_remind_2024}. \added{\citet{nissen_high-precision_2020} and \citet{nataf_accurate_2024} observed multiple sequences in the local age--metallicity relation (AMR), which could be interpreted as evidence supporting the two-infall scenario, but \citet{plotnikova_chemical_2024} argued that such patterns can arise due to selection effects.} \citet{spitoni_beyond_2023} and \citet{palla_mapping_2024} proposed a third gas accretion event in the last $\sim3$ Gyr to match the star formation history \added{inferred} from {\it Gaia} \citep{ruiz-lara_recurrent_2020} and to explain the recent abundance evolution (or lack thereof) in the Solar neighborhood. Most recently, \citet{hegedus_reconstructing_2025} constrained the infall timescales using chemical abundances for $\sim400,000$ stars from the 19th data release of the Sloan Digital Sky Survey \citep[SDSS;][]{sdss_collaboration_nineteenth_2025}.

While previous studies have explored a large parameter space, most have assumed that the Galactic disk never \added{had} a substantial outflow. This decision is based on hydrodynamic simulations of Galactic fountains \added{produced} by CC SNe, which predict that ejected material falls back onto the disk on relatively short timescales \citep{spitoni_galactic_2008,spitoni_effects_2009} close to \added{its} point of origin \citep{melioli_hydrodynamical_2008,melioli_hydrodynamical_2009}, suggesting that the effect on GCE should be minimal. However, the effects of feedback in simulations are sensitive to its implementation \citep[e.g.,][]{li_effects_2020,hu_code_2023}; other simulations of MW-like galaxies with different prescriptions do produce mass-loaded outflows \citep[e.g.,][]{brook_hierarchical_2011,gutcke_nihao_2017,nelson_first_2019,peschken_angular_2021,kopenhafer_seeking_2023}. Empirically, mass-loading \added{is} observed in nearby starburst galaxies \citep[e.g.,][]{lopez_temperature_2020,cameron_duvet_2021,lopez_x-ray_2023} but not MW-like systems because the predicted column densities are below current detection limits \citep[see reviews by][]{veilleux_cool_2020,thompson_theory_2024}. Even if the MW is not currently ejecting a substantial outflow, it may have \added{done so} in the past.

By neglecting Galactic outflows, previous studies of the two-infall scenario have been constrained in their choice of nucleosynthetic yields \citep{francois_evolution_2004} because of the yield--outflow degeneracy \citep[e.g.,][]{hartwick_chemical_1976,cooke_primordial_2022,johnson_dwarf_2023,sandford_strong_2024}. \citet{weinberg_equilibrium_2017} showed that the equilibrium metallicity is primarily set by the yields and the outflow mass-loading factor; proportionally raising or lowering both may affect the time-dependence of abundance evolution but not the end-point. This degeneracy prohibits direct estimates of the \added{yields from GCE models} unless the effect of outflows is assumed to be insignificant. The predicted yields from CCSN models can vary substantially depending on the choice of initial mass function \citep{vincenzo_modern_2016} and the physics of black hole formation \citep{griffith_impact_2021}, yet few studies have investigated the effect of \added{varying} the yield scale on two-infall scenario predictions. Varying the yield scale while maintaining an evolutionary end point \added{consistent} with observations is straightforwardly achieved with simultaneous variations in the strength of outflows.

% The observed chemical and kinematic separation of the high- and low-$\alpha$ disks remains the primary motivation for the two-infall scenario. \citet{spitoni_remind_2024} argued that the observed gap between the sequences in \aFe, contrasted with their overlap in $[\alpha/{\rm H}]$, indicates a period of reduced star formation, which is a natural consequence of the delay between the two accretion epochs. In different samples, \citet{nissen_high-precision_2020} and \citet{nataf_accurate_2024} observed multiple sequences in the local age--metallicity relation (AMR), which could be interpreted as evidence supporting the two-infall scenario. However, \citet{plotnikova_chemical_2024} found no evidence for multiple sequences in the local AMR, and argued that such patterns can arise due to selection effects. Many two-infall studies have also reproduced the metallicity gradient, the local surface densities of stars and gas, and the local star formation and supernova rates \citep[e.g.,][]{chiappini_chemical_1997,romano_mass_2000,spitoni_remind_2024}, although most GCE models in the literature benchmark against these observables \citep[e.g.,][]{minchev_chemodynamical_2013,johnson_stellar_2021}.

\begin{figure}
    \centering
    \includegraphics[width=\linewidth]{figures/smooth_vs_twoinfall.pdf}
    \caption{Chemical abundance tracks predicted by a two-infall model at $R_{\rm gal}=8\kpc$ (solid curve) versus a model with a smooth SFH at three different radii (dashed curves). The two-infall model adopts the fiducial parameters according to Table \ref{tab:parameters}, while the smooth SFH adopts the parameters of the ``inside-out'' model of \citet{johnson_stellar_2021}. Both models assume the \yZ{2} yield scale (see Table \ref{tab:yields}). The 2-D histogram shows the number of stars from APOGEE DR17 in the Solar neighborhood ($7\leq R_{\rm gal}<9\kpc$, $0\leq|z|<2\kpc$) in each bin of ([Fe/H], [O/Fe]).}
    \label{fig:smooth-vs-twoinfall}
\end{figure}

\added{The two-infall scenario reproduces} the full distribution of stellar abundances in the Solar neighborhood through a single, continuous evolutionary track. However, the current body of evidence suggests that many of the stars that make up the wings of the local metallicity distribution originate from elsewhere in the Galaxy. \citet{sellwood_radial_2002} first showed that transient spiral perturbations can induce large changes in the guiding radius of a star without kinematic heating, and it is now understood that the stars that make up the Solar neighborhood are drawn from a wide range of birth radii \citep[e.g.,][]{schonrich_chemical_2009,frankel_measuring_2018,lian_quantifying_2022,lehmann_probing_2024}. 
% Some studies have attempted to derive stellar birth radii \citep[e.g.,][]{ratcliffe_unveiling_2023,lu_there_2024}, though such an endeavor depends on a model of the evolution of the radial metallicity profile in the MW. 
% While the strength and speed of radial migration in the disk are not precisely measured, a single chemical evolution track need not explain the entire width of the local abundance distribution. 
\added{A number of GCE studies have reproduced the $\alpha$-bimdodality using a combination of radial migration and smooth, inside-out galaxy growth \citep[e.g.,][]{schonrich_chemical_2009,kubryk_evolution_2015,sharma_chemical_2021,chen_chemical_2023,prantzos_origin_2023}. In this scenario, the local high-$\alpha$ population originates from the inner Galaxy, where the star formation rate peaked early in its history (although the models of \citealt{johnson_stellar_2021} and \citealt{dubay_galactic_2024} failed to produce an $\alpha$-bimodality).}
Figure \ref{fig:smooth-vs-twoinfall} contrasts the chemical evolution track for the Solar neighborhood predicted by a two-infall model against tracks from the smooth SFH model of \citet{johnson_stellar_2021} at three different radii. 
If the effect of radial migration is significant, then the \added{local $\alpha$-bimodality could} be explained without dramatic changes to the local gas abundance over time.

% In contrast to the two-infall scenario, a number of studies have attempted to explain the $\alpha$-bimodality through purely secular processes. 
% \citet{schonrich_chemical_2009} first showed that radial migration can produce an $\alpha$-bimodality through a superposition of chemical tracks, although they did not directly compare against MW data. Others have produced a more MW-like $\alpha$-bimdodality using a combination of radial migration and inside-out galaxy growth \citep[e.g.,][]{kubryk_evolution_2015,sharma_chemical_2021,chen_chemical_2023,prantzos_origin_2023}. In this scenario, the local high-$\alpha$ population originates from the inner Galaxy, where the star formation rate peaked early in its history. \citet{sharma_chemical_2021} and \citet{chen_chemical_2023} showed that a simultaneous decline in the star formation rate and the \aFe ratio is needed to separate the sequences in chemical space. \citet{chen_recent_2025} additionally argued that the double sequence in the local age--metallicity relation observed by \citet{nissen_high-precision_2020} can also be explained by a smooth star formation history and inside-out disk growth. However, other GCE models that incorporate both radial migration and smooth, inside-out star formation have failed to predict an $\alpha$-bimodality \citep[e.g.,][]{johnson_stellar_2021,dubay_galactic_2024}. While the $\alpha$-bimodality remains a key piece of evidence for the two-infall scenario, it is not the only proposed explanation.

\added{As stellar age} estimation techniques have improved over recent years, large catalogs have become available with \added{age estimates} for stars across a wide swath of the Galaxy. In a challenge to the traditional view of GCE, which expects the ISM metallicity to continually increase over time, these catalogs \added{suggest} that stellar metallicities are remarkably consistent with age across most of the \added{thin disk} \citep[e.g.,][]{spina_mapping_2022,magrini_gaia-eso_2023,willett_evolution_2023,carbajo-hijarrubia_occaso_2024,gallart_chronology_2024}. \citet{johnson_milky_2024} demonstrated that these results are readily explained by a scenario in which the Galaxy reaches a state of chemical equilibrium. In this so-called ``equilibrium scenario,'' the local metallicity is driven by the ratio of star formation to accretion at a given radius, which evolves to a constant over Gyr timescales. Whether the equilibrium metallicity is regulated by outflows, as in the  \citet{johnson_milky_2024} models, or by other factors such as a radial gas flow \citep{spitoni_effects_2011,bilitewski_radial_2012,sharda_interplay_2024}, the current data suggest that metal abundances in the ISM evolved minimally during the \added{past $\sim8\Gyr$}.

\added{The availability of} large stellar catalogs with [Fe/H], \aFe, {\it and} age covering a large portion of the Galactic disk enables much more comprehensive tests of the two-infall GCE scenario (and other models). Our study draws on the steadiness of the Milky Way stellar metallicity gradient found by \citet{johnson_milky_2024} and empirical constraints on the yield scale from \citet{weinberg_scale_2024}. We run multi-zone GCE models with a two-infall accretion history, radially-dependent mass-loaded outflows, and a prescription for radial migration tuned to a hydrodynamical simulation. We investigate the impact of the scale of SN yields and outflows, the strength of radial migration, the composition of the circumgalactic medium, and the local disk mass surface density ratio on GCE model predictions. We compare our results to abundance distributions across the disk from APOGEE DR17, and to age--abundance relations from multiple catalogs. We describe our observational sample in Section \ref{sec:observational-sample}, and we detail our chemical evolution models and parameter selection in Section \ref{sec:methods}. We compare our multi-zone model predictions to the data in Section \ref{sec:multizone-results}. We discuss possible extensions to our models in Section \ref{sec:discussion}, and we and summarize our conclusions in Section \ref{sec:conclusions}.

\section{Observational Sample}
\label{sec:observational-sample}

\begin{table*}
    \centering
    \caption{Sample selection parameters from APOGEE DR17 (see Section \ref{sec:observational-sample}).}
    \label{tab:sample}
    \begin{tabular}{lll}
        \hline\hline
        \multicolumn{1}{c}{Parameter} & \multicolumn{1}{c}{Range or Value} & \multicolumn{1}{c}{Notes} \\
        \hline
        $\log g$            & $1.0 < \log g < 3.8$          & Select giants only \\
        $T_{\rm eff}$       & $3500 < T_{\rm eff} < 5500$ K & Reliable temperature range \\
        $S/N$               & $S/N > 80$                    & Required for accurate stellar parameters \\
        ASPCAPFLAG Bits     & $\notin$ 23                   & Remove stars flagged as bad \\
        EXTRATARG Bits      & $\notin$ 0, 1, 2, 3, or 4     & Select main red star sample only \\
        $R_{\rm gal}$     & $3 < R_{\rm gal} < 15\kpc$    & Eliminate bulge \& extreme outer-disk stars \\
        $|z|$               & $|z| < 2\kpc$                 & Eliminate halo stars \\
        \hline
        NN age error        & $\sigma_\tau/\tau < 40\%$     & Age uncertainty from \citet{leung_variational_2023} \\
        \added{Uncertainty on [C/N]}    & $\sigma_{\rm [C/N]}\leq0.1$     & Precise [C/N] ages from \citet{roberts_cn_2025} \\
        \added{${\rm [Fe/H]}$}      & $\mathFeH\geq-0.4$            & Accurate [C/N] ages for upper RGB \& RC stars \\
        \hline
    \end{tabular}
\end{table*}

\begin{table}
    \centering
    \caption{Median and dispersion in APOGEE parameter uncertainties.}
    \label{tab:uncertainties}
    \begin{tabular}{l|cc}
\hline\hline
 & Median & Uncertainty  \\
Parameter & Uncertainty & Dispersion ($95\% - 5\%$) \\
\hline
${\rm [O/H]}$ & 0.019 & 0.031 \\
${\rm [Fe/H]}$ & 0.0089 & 0.006 \\
$\log_{10}(\tau_{\rm NN}/{\rm Gyr})$ & 0.1 & 0.16 \\
$\tau_{\rm [C/N]}/{\rm Gyr}$ & 1.4 & 1.8 \\
\hline
\end{tabular}

\end{table}

We compare our models against stellar abundances from the Apache Point Observatory Galactic Evolution Experiment \citep[APOGEE;][]{majewski_apache_2017} data release 17 \citep[DR17;][]{abdurrouf_seventeenth_2022}. APOGEE data were obtained from infrared spectrographs \citep{wilson_apache_2019} mounted on the 2.5-meter Sloan Foundation Telescope \citep{gunn_25_2006} at Apache Point Observatory and the Ir{\'e}n{\'e}e DuPont Telescope \citep{bowen_optical_1973} at Las Campanas Observatory. The data reduction pipeline is described by \citet{nidever_data_2015}, and \added{the} APOGEE Stellar Parameter and Chemical Abundance Pipeline (ASPCAP) is detailed by \citet{holtzman_abundances_2015}, \citet{garcia_perez_aspcap_2016}, and \citet{jonsson_apogee_2020}.

We obtain a sample of \num{171635} red giant branch and red clump stars with high-quality spectra using the selection criteria listed in Table \ref{tab:sample}, which are adapted from \citet{hayden_chemical_2015}. Table \ref{tab:uncertainties} presents the median statistical uncertainty and uncertainty dispersion ($95^{\rm th} - 5^{\rm th}$ percentile difference) of the calibrated [Fe/H] and [O/Fe]\footnote{
    In this paper, we use O as the representative $\alpha$-element in our GCE models and stellar abundance data. We refer to [O/Fe] and \aFe interchangeably, though observational studies often average several $\alpha$-element abundances to calculate \aFe.
} abundances for our sample. When calculating the galactocentric radius $R_{\rm gal}$ and midplane distance $z$ of each star, we use the \citet{bailer-jones_estimating_2021} photo-geometric distance estimates from {\it Gaia} Early Data Release 3 \citep{gaia_collaboration_gaia_2016,gaia_collaboration_gaia_2021} included in the APOGEE DR17 catalog and we adopt the Galactic coordinates of the Sun $(R,z)_\odot=(8.122, 0.0208)\kpc$ \citep{gravity_collaboration_detection_2018,bennett_vertical_2019}.

\subsection{Stellar Age Estimates}
\label{sec:age-estimates}

We supplement the APOGEE DR17 abundance data with two different age catalogs. The first is from \citet{leung_variational_2023}, who \added{trained} a variational encoder-decoder network on asteroseismic data for APOGEE red giants with $2.5<\log g<3.6$. 
% This catalog has two main advantages over other neural network (NN) age estimates: their method is designed to reduce contamination from abundance information (in particular [$\alpha$/Fe]), and they recover accurate ages up to $\sim13\Gyr$, while other neural network-derived age catalogs exhibit an age plateau \citep[e.g.,][]{mackereth_dynamical_2019}. 
\added{Following} the recommendations of \citet{leung_variational_2023}, we cut all stars which have an age uncertainty greater than 40\%. This produces a sample of \num{57607} stars with NN age estimates, of which \num{14871} are in the Solar neighborhood ($7\leq R_{\rm gal}<9\kpc$, $0\leq|z|<0.5\kpc$). The median uncertainty in $\log({\rm age})$ is 0.10 (see Table \ref{tab:uncertainties}), although the oldest stars typically have smaller uncertainties.

Our second age catalog utilizes the [C/N]--age relation calibrated by \citet{roberts_cn_2025} for red giant branch (RGB) and red clump \added{(RC)} stars. The relationship relies on the mass-dependent level of mixing during first dredge-up \citep[FDU;][]{iben_stellar_1967} to map the correlation of stellar mass, and hence age, with surface chemistry. This method has the benefit of providing age estimates for luminous giants ($\log g<2.5$), which increases the sample size at larger distances from the Sun. \added{Uncertainties in} the efficiency of FDU mixing and the RGB age--mass relationship mean \added{that} the ages are not trustworthy outside the range $1\sim10\Gyr$, but this is the age range most useful for our purposes anyway. Additional mixing effects in low-metallicity stars also prevent the relation from being applied to luminous \added{RGB and RC} stars with $\mathFeH<-0.4$; stars on the lower RGB do not suffer from this problem. 
% The median propagated uncertainty for the [C/N]-derived ages is $\sim1\Gyr$; however, as noted by \citet[][submitted]{roberts_cn_2025}, the propagated errors underestimate the true age uncertainty, so we enhance the uncertainties by 40\% (see Table \ref{tab:uncertainties}).
\added{We use the sample selection criteria from \citet{roberts_cn_2025}, and following their recommendation}, we adopt a flat $1.64\Gyr$ age uncertainty for all stars.
With this relationship, we estimate ages for \num{124778} stars across the disk, including \num{21956} in the Solar neighborhood.

\section{Chemical Evolution Models \& Parameter Selection}
\label{sec:methods}

We run multi-zone GCE models using the Versatile Integrator for Chemical Evolution \citep[{\tt VICE};][]{johnson_impact_2020}. The basic format of our models follows \citet{johnson_stellar_2021} and \citet{dubay_galactic_2024}. We set up a disk \added{spanning} $0\leq R_{\rm gal}<20\kpc$ that is divided into concentric rings of width $\delta R_{\rm gal}=100\,{\rm pc}$. We use a time-step of $\Delta t=10\,{\rm Myr}$, $n=8$ stellar populations per time-step per ring, and we run our models to a final time of $t_{\rm final}=13.2\,{\rm Gyr}$. Within each ring, chemical evolution proceeds according to a conventional one-zone GCE model with instantaneous mixing and continuous recycling. Stellar populations migrate between zones, allowing the long-lived progenitors of SNe Ia to enrich areas of the Galaxy outside of their birth zones, which couples the enrichment in nearby rings. We inhibit star formation past $R_{\rm gal}>15.5\kpc$, so stars in the outer 4.5 kpc of the model disk represent a purely migrated population.

We discuss our assumptions about the nucleosynthetic yields in Section \ref{sec:yields}, the outflow prescription in Section \ref{sec:outflows}, the star formation law in Section \ref{sec:sf-law}, the gas supply equation in Section \ref{sec:sfh}, 
% the infall parameter selection in Section \ref{sec:parameter-selection}, 
\added{and} the stellar migration prescription in Section \ref{sec:migration}. We do not incorporate radial gas flows between the different zones, but we discuss their potential implications in Section \ref{sec:radial-flows}. Table \ref{tab:parameters} summarizes the model parameters and their fiducial values.

\begin{deluxetable*}{Ccccl}
    \tablecaption{A summary of variables and their fiducial values for our chemical evolution models (see discussion in Section \ref{sec:methods}).\label{tab:parameters}}
    \tablehead{
        \colhead{Quantity} & \colhead{Fiducial Value} & \colhead{Alternatives} & \colhead{Section} & \colhead{Description}
    }
    \startdata
        R_{\rm gal}         & $[0,20]\kpc$  & \nodata           & \ref{sec:methods}     & Galactocentric radius \\
        \delta R_{\rm gal}  & $0.1\kpc$     & \nodata           & \ref{sec:methods}     & Radial zone width \\
        R_{\rm SF}          & $15.5\kpc$    & \nodata           & \ref{sec:methods}     & Maximum radius of star formation \\
        t_{\rm final}       & $13.2\Gyr$    & \nodata           & \ref{sec:methods}     & Disk lifetime \\
        \Delta t            & $10\Myr$      & \nodata           & \ref{sec:methods}     & Time-step size \\
        n                   & 8             & \nodata           & \ref{sec:methods}     & Number of stellar populations formed per ring per timestep \\
        \dot M_r            & continuous    & \nodata           & \ref{sec:methods}     & Recycling rate \citep[][Equation 2]{johnson_impact_2020} \\
        y/Z_\odot           & 1             & 2, 3              & \ref{sec:yields}      & Scale of nucleosynthetic yields (see Table \ref{tab:yields}) \\
        N_{\rm Ia}/M_\star  & $1.55\times10^{-3}\Msun^{-1}$ & 2.62, $3.57\times10^{-3}\Msun^{-1}$   & \ref{sec:yields}  & SNe Ia per unit mass of stars formed \\
        f_{\rm Ia}(t)       & Equation \ref{eq:plateau-dtd} & Equation \ref{eq:powerlaw-dtd}    & \ref{sec:yields}  & Delay-time distribution of Type Ia supernovae \\
        t_{\rm Ia}          & $40\Myr$  & \nodata           & \ref{sec:yields}              & Minimum SN Ia delay time \\
        \eta_\odot          & 0.2       & $0.6-2.4$         & \ref{sec:outflows}            & Outflow mass-loading factor at $R_\odot$ ($\eta\equiv\dot\Sigma_{\rm out}/\dot\Sigma_\star$) \\
        R_\eta              & $5.0\kpc$ & \nodata           & \ref{sec:outflows}            & Exponential outflow scale radius \\
        \tau_\star          & Equation \ref{eq:sf-law}      & \nodata & \ref{sec:sf-law}      & Star formation efficiency timescale ($\tau_\star\equiv\Sigma_g/\dot\Sigma_\star$) \\
        k                   & 1.5       & \nodata           & \ref{sec:sf-law}              & Star formation law exponent \citep{kennicutt_global_1998} \\
        M_{\rm \star,tot}   & $5.17\times 10^{10}\Msun$     & \nodata & \ref{sec:sfh} & Total stellar mass of the disk \citep{licquia_improved_2015} \\
        f_\Sigma(R_\odot)   & 0.12      & 0.25, 0.5         & \ref{sec:sfh}                 & Local thick/thin disk surface density ratio \\
        R_1                 & $2.0\kpc$ & \nodata           & \ref{sec:sfh}                 & Thick disk scale radius \\
        R_2                 & $2.5\kpc$ & \nodata           & \ref{sec:sfh}                 & Thin disk scale radius \\
        \mathXH_{\rm CGM}   & $-\infty$ & $-0.7$, $-0.5$    & \ref{sec:sfh}                 & Metallicity of infalling gas \\
        f_{\rm in}(t|R_{\rm gal})   & Equation \ref{eq:infall-rate} & \nodata   & \ref{sec:sfh} & Time-dependence of the gas accretion rate \\
        \tau_1              & $1\Gyr$   & \nodata           & \ref{sec:sfh} & Timescale of the first infall epoch \\
        \tau_2(R_\odot)     & $15\Gyr$  & \nodata           & \ref{sec:sfh} & Timescale of the second infall epoch at the Solar annulus \\
        R_{\tau_2}          & $7\kpc$   & \nodata           & \ref{sec:sfh} & Exponential scale radius of the second infall timescale \\
        t_{\rm max}         & $4.2\Gyr$ & \nodata           & \ref{sec:sfh} & Time of maximum gas infall (onset of second infall) \\
        \sigma_{\rm RM}(\tau,R_{\rm form})  & Equation \ref{eq:radial-migration}    & \nodata & \ref{sec:migration} & Width of Gaussian controlling radial migration distance \\
        \sigma_{\rm RM8}    & $2.68\kpc$ & $3.6$, $5.0\kpc$ & \ref{sec:migration}   & Radial migration strength (width of $\sigma_{\rm RM}$ for $\tau=8\Gyr$) \\
        z                   & $[-3,3]\kpc$  & \nodata       & \ref{sec:migration}   & Distance from Galactic midplane at present day \\
        h_z(\tau,R_{\rm final}) & Equation \ref{eq:scale-height}    & \nodata       & \ref{sec:migration} & Disk scale height \\
    \enddata
\end{deluxetable*}
\vspace{-24pt}

\subsection{Nucleosynthetic Yields}
\label{sec:yields}

\begin{table}
    \centering
    \caption{Nucleosynthetic yields at each of the yield scales (see Section \ref{sec:yields}).}
    \begin{tabular}{c|cc}
\hline\hline
Element & $y_{\rm X}^{\rm CC}$ & $y_{\rm X}^{\rm Ia}$ \\
\hline
O & \num{5.91e-03} & \num{0} \\
Mg & \num{7.29e-04} & \num{0} \\
Si & \num{5.58e-04} & \num{1.44e-04} \\
Fe & \num{4.73e-04} & \num{7.74e-04} \\
\hline
\end{tabular}

    \label{tab:yields}
\end{table}

The population-averaged nucleosynthetic yields of CCSNe, $y_{\rm X}^{\rm CC}$, are uncertain to a degree that is significant for chemical evolution models. 
% This problem is exacerbated by the complexity of the CCSN explosion landscape \citep{sukhbold_core-collapse_2016}. Recently, 
\added{\citet{weinberg_scale_2024} used} a measurement of the mean Fe yield of CC SNe by \citet{rodriguez_iron_2023} and the plateau in stellar \aFe abundances at low metallicity to infer population-averaged yields of $y/Z_\odot\approx1$\footnote{
    In this work, we will use the $y/Z_\odot$ notation to refer to the scale set by the massive star $\alpha$-element yields; i.e., $y_{\alpha}^{\rm CC}/Z_{\alpha,\odot}$. We also clarify that these yields refer to the net metal production by stellar populations; the return of previously produced metals in the envelopes of dying stars is handled separately by {\tt VICE}.
}\textemdash in other words, for every $1\Msun$ of stars formed, massive stars release a \added{quantity} of newly-synthesized $\alpha$-elements (e.g., O or Mg) equal to their mass \added{fraction} in the Sun. However, \citet{johnson_milky_2024} found that their GCE models with \added{this yield} scale approach present-day abundances too slowly to match the observed age--metallicity relation. Previous multi-zone models using {\tt VICE} \citep[e.g.,][]{johnson_stellar_2021,dubay_galactic_2024} adopted higher yields ($y/Z_\odot\approx2.6$) based on \citet{chieffi_explosive_2004} and \citet{limongi_nucleosynthesis_2006}; however, in order to produce a realistic evolution of [O/Fe], those studies adopted an integrated SN Ia rate which is high compared to the measurement of \citet{maoz_star_2017}.

We therefore investigate yield sets at multiple scales. The CCSN yield of O is directly set by the Solar scale, $y_{\rm O}^{\rm CC}=(y/Z_\odot)Z_{\rm O,\odot}$\footnote{
    We adopt the \citet{asplund_chemical_2009} Solar abundances: $Z_{\rm O,\odot}=5.72\times10^{-3}$ and $Z_{\rm Fe,\odot}=1.29\times10^{-3}$. We note that for this empirical scaling, choosing different Solar abundances \citep[e.g.,][]{magg_observational_2022} would lead us to change the yields proportionally, so that our GCE model predictions in Solar-scaled abundances would actually be unchanged.
}, because we assume that all O is produced by CCSNe. For Fe, the CCSN yield is set by the \aFe ``plateau'' at low metallicity, $\mathOFe_{\rm CC}$, such that $y_{\rm Fe}^{\rm CC}=(y/Z_\odot) Z_{\rm Fe,\odot} 10^{-\mathOFe_{\rm CC}}$ \citep[for further discussion on the empirical yield scale and the CCSN plateau, see][]{weinberg_scale_2024}. We set the plateau at $\mathOFe_{\rm CC}=+0.45$, which corresponds to an Fe yield from CCSNe of $y_{\rm Fe}^{\rm CC}=4.58\times10^{-4}(y/Z_\odot)$. Our yield sets are presented in Table \ref{tab:yields}. We consider \yZ{1} representative of the empirical yield scale, whereas \yZ{2-3} span a range of theoretical predictions.

The SN Ia yield of Fe, $y_{\rm Fe}^{\rm Ia}$, is set so that our models reach $\mathOFe\approx0.0$ by $t=13.2\,{\rm Gyr}$. For \yZ{3}, the combined Fe yield of CCSNe and SNe Ia matches the Solar yield scale: $(y_{\rm Fe}^{\rm Ia}+y_{\rm Fe}^{\rm CC})/Z_{\rm Fe,\odot}=y/Z_\odot$; for \yZ{1} and \yZ{2}, we enhance $y_{\rm Fe}^{\rm Ia}$ by a factor of 30\% and 10\%, respectively, to reach the desired end-point. The fifth row of Table \ref{tab:yields} reports the integrated SN Ia rate or total number of SNe Ia per unit mass of star formation,
\begin{equation}
    \frac{N_{\rm Ia}}{M_\star} = \frac{y_{\rm Fe}^{\rm Ia}}{\bar m_{\rm Fe}^{\rm Ia}},
    \label{eq:snia-rate}
\end{equation}
\added{for} each yield set, assuming a mean Fe yield per SN Ia of $\overline m_{\rm Fe}^{\rm Ia}=0.7$ M$_\odot$ \citep{mazzali_common_2007,howell_effect_2009}. The rate for the \yZ{1} yield set is slightly higher than the volumetric rate of $N_{\rm Ia}/M_\star=(1.3\pm0.1)\times10^{-3}\,{\rm M}_\odot^{-1}$ reported by \citet{maoz_star_2017}, but is consistent with their measurement of $N_{\rm Ia}/M_\star=(1.6\pm0.3)\times10^{-3}\,{\rm M}_\odot^{-1}$ for field galaxies. The rate for the \yZ{2} yield set is consistent with the measurement of $N_{\rm Ia}/M_\star=(2.2\pm1.0)\times10^{-3}\,{\rm M}_\odot^{-1}$ by \citet{maoz_type-ia_2012}, while the rate for the \yZ{3} yield set is generally higher than literature values.

Unlike CCSNe, SNe Ia populate a broad distribution of delay times between progenitor formation and explosion. The time-dependent SN Ia rate per unit mass of star formation is defined as
\begin{equation}
    R_{\rm Ia}(t) = 
    \begin{cases}
        \frac{N_{\rm Ia}}{M_\star}
        \frac{f_{\rm Ia}(t)}{\int_{t_{\rm Ia}}^{t_{\rm final}} f_{\rm Ia}(t') dt'}, & t \ge t_{\rm Ia} \\
        0 & t < t_{\rm Ia},
    \end{cases}
    \label{eq:dtd-function}
\end{equation}
where $t_{\rm Ia}=40\,{\rm Myr}$ is the minimum SN Ia delay time, $t_{\rm final}=13.2\,{\rm Gyr}$ is our assumed lifetime of the Galactic disk, and $f_{\rm Ia}(t)$ is the un-normalized form of the DTD. Motivated by recent results suggesting that a large fraction of long-delayed SNe Ia improves agreement with the Milky Way's high-$\alpha$ sequence \citep{palicio_analytic_2023,dubay_galactic_2024}, we adopt a wide plateau DTD of the form
\begin{equation}
    \label{eq:plateau-dtd}
    f_{\rm Ia}(t) =
    \begin{cases}
        1, & t < 1\,{\rm Gyr} \\
        (t/1\,{\rm Gyr})^{-1.1}, & t \ge 1\,{\rm Gyr}.
    \end{cases}
\end{equation}
We also explore a simple power-law DTD,
\begin{equation}
    f_{\rm Ia}^{\rm plaw}(t) = (t/1\,\rm{Gyr})^{-1.1}.
    \label{eq:powerlaw-dtd}
\end{equation}
Equation \ref{eq:powerlaw-dtd} is consistent with SN Ia surveys \citep[e.g.,][]{maoz_star_2017} and results in more prompt Fe production.
We discuss the implications of using the power-law DTD in Section \ref{sec:abundance-distributions}.

Many previous two-infall studies have adopted the yields of \citet{francois_evolution_2004}, who in turn adapted those of \citet{woosley_evolution_1995} for CCSNe and \citet{iwamoto_nucleosynthesis_1999} for SNe Ia to provide a better fit between GCE models and local abundance data. Notably, the yields for O and Fe were left unchanged from the original studies. However, because \citet{woosley_evolution_1995} report gross yields without detailed initial abundances for their CCSN progenitors, and because \citet{francois_evolution_2004} do not provide population-averaged yields, it is difficult to make a comparison with our yields. Ultimately, \citet{francois_evolution_2004} report that their GCE models are insensitive to changes in the CCSN yield of O by a factor of 2, so we consider it reasonable to explore the full range of yields given in Table \ref{tab:yields}.

\subsection{Outflows}
\label{sec:outflows}

Mass-loaded outflows are a useful tool for scaling the endpoint of GCE models. \citet{weinberg_equilibrium_2017} showed that in the case of exponentially declining star formation, the O abundance approaches an equilibrium at
\begin{equation}
    \label{eq:equilibrium}
    Z_{\rm O,eq} = \frac{y_{\rm O}^{\rm CC}}{1 + \eta - r - \tau_\star/\tau_{\rm SFH}},
\end{equation}
where $r=0.4$ is the instantaneous recycling parameter, $\tau_\star$ is the star formation efficiency timescale, $\tau_{\rm SFH}$ is the $e$-folding timescale of the star formation history, and $\eta\equiv \dot\Sigma_{\rm out}/\dot\Sigma_\star$ is the outflow mass-loading factor. Motivated by Equation \ref{eq:equilibrium}, we adopt a different outflow mass-loading factor at the Solar radius $\eta_\odot\equiv\eta(R=R_\odot)$ for each of the yield sets in Table \ref{tab:yields}. 
\added{Measurements of gas-phase \citep[e.g.,][]{mendez-delgado_gradients_2022} and stellar (Figure \ref{fig:yield-outflow}) abundances indicate that the Solar neighborhood ISM is presently close to Solar metallicity, so we adjust $\eta_\odot$ to ensure that $Z_{\rm O,eq}\approx Z_{\rm O,\odot}$.}
% Models with lower yields do not achieve a steady-state abundance in time (see Figure \ref{fig:yield-outflow}); therefore, the values of $\eta_\odot$ for \yZ{1} and \yZ{2} are lower than Equation \ref{eq:equilibrium} would indicate as required to reach Solar metallicity at the present day, assuming that $Z_{\rm O,eq}\approx Z_{\rm O,\infty}$ as suggested by \citet{johnson_milky_2024} based on stellar age trends. 
Table \ref{tab:models} reports the value of $\eta_\odot$ in each of our models.

% Not all GCE studies have constrained their models to reach an equilibrium at the Solar metallicity. For example, the models of \citet{palla_chemical_2020} and \citet{spitoni_remind_2024} predict somewhat super-Solar metallicity in the present-day Solar neighborhood. Measurements of gas-phase \citep[e.g.,][]{mendez-delgado_gradients_2022} and stellar (Figure \ref{fig:yield-outflow}) abundances  indicate that the Solar neighborhood ISM is presently close to Solar metallicity, so we use $\eta$ to fine-tune the chemical evolution end-point to $\mathOH\approx0.0$.

% Equation \ref{eq:equilibrium} suggests that one can achieve a different $Z_{\rm O,eq}$ in different regions of the Galaxy by adopting a spatially-varying prescription for $\eta$. \citet{johnson_milky_2024} used this approach to present a proof of concept for the equilibrium scenario. 
\added{\citet{johnson_milky_2024} used variations in $\eta$ with $R_{\rm gal}$ to produce variations in $Z_{\rm O,eq}$.
To} produce an exponentially declining radial \added{[O/H]} gradient, we adopt a prescription of
\begin{equation}
    \eta(R_{\rm gal}) = \eta_\odot \exp\Big(\frac{R_{\rm gal}-R_\odot}{R_\eta}\Big)
\end{equation}
\added{for the outflow mass-loading factor,}
where $R_\eta$ is the exponential outflow scale radius and $R_\odot=8\kpc$. We adopt $R_\eta=5$ kpc, a lower value than in \citet{johnson_milky_2024}, so that our $y/Z_\odot=1$ model produces a radial abundance gradient of $\nabla\mathOH_{\rm eq}\approx-0.06\,{\rm dex}\kpc^{-1}$, \added{consistent} with recent measurements from HII regions \citep{mendez-delgado_gradients_2022} and stars \citep{myers_open_2022,johnson_milky_2024}. \added{We note, however, that mass-loaded outflows are not a necessary ingredient for the results of this study. We find that a one-zone model with the fiducial parameters, $\eta=0$, and \yZ{0.8}, predicts a similar abundance evolution and nearly identical stellar abundance distributions to the fiducial model with $\eta=0.2$ and \yZ{1}.}

% Most previous studies of the two-infall model have assumed that the Milky Way has experienced no significant mass-loaded outflows. Even in studies that do incorporate Galactic winds, the mass-loading is relatively weak \citep[e.g., $\eta\approx0.2$ in][]{palicio_analytic_2023} or does not vary with radius \citep{hegedus_reconstructing_2025}. To achieve a realistic radial metallicity gradient, many studies have adopted the yields of \citet{francois_evolution_2004} and a prescription for the infall timescale of the thin disk that increases linearly with radius \citep[e.g.,][]{chiappini_chemical_1997,romano_mass_2000}. Additionally, some studies have implemented radial gas flows or a variable star formation efficiency in order to regulate the radial metallicity gradient \citep[e.g.,][]{spitoni_effects_2011,palla_chemical_2020}.

% As discussed by \citet{johnson_milky_2024}, evidence for or against outflows in Milky Way-type galaxies in simulations and observations is inconclusive. Here, we aim to study the effect of the assumed yield scale on two-infall model predictions, so we use mass-loaded outflows to control the final state of chemical evolution across the disk. However, mass-loaded outflows are not a necessary ingredient for the results of this study. We find that a one-zone model with the fiducial parameters, $\eta=0$, and \yZ{0.8}, predicts a similar abundance evolution and nearly identical stellar abundance distributions to the fiducial model with $\eta=0.2$ and \yZ{1}.

\subsection{The Star Formation Law}
\label{sec:sf-law}

\added{We use a power-law star formation prescription of} $\dot\Sigma_\star\propto\Sigma_g^k$, with $k=1.5$ following \citet{kennicutt_global_1998}. Previous work with this GCE model \citep[e.g.,][]{johnson_stellar_2021,dubay_galactic_2024} assumed a three-component power-law, but we adopt a single \added{power-law to allow} for a more direct comparison with previous two-infall studies \citep[e.g.,][]{spitoni_remind_2024}. In detail, we calculate the star formation efficiency (SFE) timescale
\begin{equation}
    \label{eq:sf-law}
    \tau_\star \equiv \frac{\Sigma_g}{\dot\Sigma_\star} = 
    \begin{cases}
        \varepsilon(t) \tau_{\rm mol}(t),   & \Sigma_g \ge \Sigma_{g,0} \\
        \varepsilon(t) \tau_{\rm mol}(t) \Big(\frac{\Sigma_g}{\Sigma_{g,0}}\Big)^{-1/2}, & \Sigma_g < \Sigma_{g,0},
    \end{cases}
\end{equation}
\added{with} $\Sigma_{g,0} = 10^8\,{\rm M}_\odot\kpc^{-2}$ and $\tau_{\rm mol}(t)=\tau_{\rm mol,0}(t/t_0)^\gamma$, \added{where $\gamma=1/2$, $t_0=13.8\,{\rm Gyr}$ and $\tau_{\rm mol,0}=2\,{\rm Gyr}$ are observationally calibrated} \citep{leroy_star_2008}. Previous two-infall studies \citep[e.g.,][]{spitoni_galactic_2019,spitoni_galactic_2020,palla_chemical_2020} have adopted a higher SFE during the first infall epoch than the second, which we emulate through a pre-factor
\begin{equation}
    \label{eq:sfe-prefactor}
    \varepsilon(t) = 
    \begin{cases}
        0.5, & t < t_{\rm max} \\
        1.0, & t \ge t_{\rm max}.
    \end{cases}
\end{equation}
A lower value of $\varepsilon(t<t_{\rm max})$ leads to more efficient star formation, and therefore more rapid enrichment, during the first infall epoch. However, the pre-factor has virtually no effect on the overall [O/Fe] distribution because the model is normalized to produce the same thick-to-thin-disk mass ratio regardless of the details of the star formation law. \added{We find} $\epsilon(t<t_{\rm max})=0.2$ or $1.0$ produce similar results in one-zone models. To guard against over-correcting the SFE in the early Galaxy, \added{we tested} eliminating either $\varepsilon(t)$ or $\tau_{\rm mol}(t)$ from our SFE prescription in multi-zone models and found no substantial \added{differences}.

\subsection{The Gas Supply}
\label{sec:sfh}

We run {\tt VICE} in ``infall mode,'' where we specify the gas infall density $\dot\Sigma_{\rm in}$ and the star formation efficiency (SFE) timescale $\tau_\star\equiv \Sigma_g / \dot\Sigma_\star$ as functions of time at each radius. The gas surface density $\Sigma_g$ and star formation rate $\dot\Sigma_\star$ are calculated from these two inputs as a natural outcome of the time-stepping solution, assuming zero initial gas mass in all zones.

The infall rate as a function of time and galactocentric radius can generically be described by
\begin{equation}
    \label{eq:infall-rate}
    \dot\Sigma_{\rm in}(t,R_{\rm gal}) = A f_{\rm in}(t|R_{\rm gal}) g(R_{\rm gal}),
\end{equation}
where $g(R_{\rm gal})=\Sigma_\star(R_{\rm gal}) / \Sigma_\star(R_{\rm gal}=0)$ is the stellar density gradient, $f_{\rm in}$ is the infall rate over time, and $A$ is a normalization constant. Because we incorporate mass-loaded outflows, $A$ is not analytically solvable, so first we numerically integrate the star formation rate $\dot\Sigma_\star(t,R_{\rm gal})$ and then follow the procedure outlined in Appendix B of \citet{johnson_stellar_2021} to calculate $A$. In short, the infall rate is normalized to produce a total disk stellar mass of $M_{\rm \star,tot}=(5.17\pm1.11)\times 10^{10}\,{\rm M}_\odot$ \citep{licquia_improved_2015} and to match the stellar surface density gradient of \citet{bland-hawthorn_galaxy_2016}.

\added{The infall rate} is described by two successive, exponentially declining bursts in time. The first infall component \added{forms} the thick disk, and the second produces the thin disk. \added{The un-normalized} form of the infall rate is
\begin{multline}
    \label{eq:twoinfall-ifr}
    f_{\rm in}(t|R_{\rm gal}) \propto e^{-t/\tau_1} + \\ H(t-t_{\rm max}) f_{2/1} (R_{\rm gal}) \exp\Big(\frac{-(t-t_{\rm max})}{\tau_2(R_{\rm gal})}\Big),
\end{multline}
\begin{equation*}
    H(x) \equiv 
    \begin{cases}
        1, & x \ge 0 \\
        0, & x < 0,
    \end{cases}
\end{equation*}
where $\tau_1$ and \added{$\tau_2(R_{\rm gal})$} are the first and second infall timescales, respectively, $t_{\rm max}$ is the onset of the second infall, $f_{2/1}$ is the ratio of the second infall amplitude to the first, and $H$ is the Heaviside step function. \added{We vary $\tau_2$ exponentially with $R_{\rm gal}$ and adopt a scale radius $R_{\tau_2}=7\kpc$ to match the stellar age gradients in MW-like spirals observed by \citet{sanchez_spatially_2020}.} We numerically calculate $f_{2/1}$ for each zone such that the resulting stellar \added{disk has a} surface density ratio of the thick and thin disks \added{of}
\begin{equation}
    f_\Sigma(R) \equiv \frac{\Sigma_1(R)}{\Sigma_2(R)} = f_\Sigma(R_\odot) e^{(R-R_\odot)\cdot(1/R_2 - 1/R_1)}.
\end{equation}
We adopt a thick disk scale radius of $R_1=2.0\kpc$, a thin disk scale radius of $R_2=2.5\kpc$, and a fiducial value for the local surface density ratio of $f_\Sigma(R_\odot)=0.12$ \citep{bland-hawthorn_galaxy_2016}. 

% Infall rate parameter selection
\added{Previous studies have adopted a wide range of parameter values for Equation \ref{eq:twoinfall-ifr}. After exploring the effect of the infall timescales in one-zone models, we adopt $\tau_1=1\Gyr$ and $\tau_2(R_\odot)=15\Gyr$, in line with the parameters adopted by \citet{spitoni_galactic_2020}, but longer than in \citet{nissen_high-precision_2020}, \citet{spitoni_apogee_2021}, and \citet{hegedus_reconstructing_2025}. We also adopt $t_{\rm max}=4.2\Gyr$ (i.e., a lookback time of $9\Gyr$) so that our models are compatible with the median stellar age of the thick disk of $9.14\pm0.05\Gyr$ \citep{pinsonneault_apokasc-3_2025}. \citet{spitoni_galactic_2019} discuss the effect of these parameters on the GCE model predictions in detail.}

\begin{figure}
    \centering
    \includegraphics[width=\onecolumn]{figures/star_formation_history.pdf}
    \caption{The evolution of (a) the infall surface density, (b) the star formation surface density, (c) the gas surface density, and (d) the star formation efficiency timescale as a function of time for our fiducial multi-zone model with $y/Z_\odot=1$. Each panel plots the evolution in six different zones of width $\delta R_{\rm gal}=0.1\kpc$, color-coded by Galactocentric radius.}
    \label{fig:sfh}
\end{figure}

Figure \ref{fig:sfh} plots the star formation history for several different zones from our fiducial model with $y/Z_\odot=1$. In the inner Galaxy, the infall rate $\dot\Sigma_{\rm in}$ is similar at the start of the first and second infall epochs, and the star formation rate peaks at $t\approx7\,{\rm Gyr}$. In the outer Galaxy, the infall rate at $t_{\rm max}$ is significantly higher than at $t=0$, and the star formation rate is highest at the present day. The star formation efficiency timescale $\tau_\star$ spikes near $t=0$ and $t_{\rm max}$, but otherwise increases throughout the model's duration, reaching a \added{present-day value} of $\tau_\star\approx2\,{\rm Gyr}$ in the inner disk and $\tau_\star\approx 9\,{\rm Gyr}$ in the outer disk.

The thick-to-thin disk density ratio is especially important for our GCE models as it controls the quantity of gas accreted during each infall epoch. Our fiducial value of $f_\Sigma(R_\odot)=0.12$ is on the low end of literature estimates, which range from $f_\Sigma(R_\odot)\sim0.06-0.6$ \citep[e.g.,][]{gilmore_new_1983,siegel_star_2002,juric_milky_2008,mackereth_age-metallicity_2017,fuhrmann_local_2017}. Previous two-infall studies have adopted a similarly broad range of values (e.g., $f_\Sigma(R_\odot)=0.18$ from \citealt{spitoni_apogee_2021}; $f_\Sigma(R_\odot)=0.4$ from \citealt{spitoni_remind_2024}). We therefore explore values up to $f_\Sigma(R_\odot)=0.5$ in our multi-zone models in Section \ref{sec:multizone-results}.

In most of our models, we assume the infalling gas is pristine (i.e., $Z_{\rm in}=0$). However, \added{previous GCE studies suggest that some level of enrichment of the infalling gas can improve agreement with observations \citep[e.g.,][]{palla_chemical_2020,johnson_milky_2024,spitoni_remind_2024}.} The circumgalactic medium (CGM) from which the infalling gas is drawn could be previously enriched, possibly from contributions from Galactic outflows, gas stripped from dwarf galaxies, or from SNe in the halo. The Milky Way's CGM is diffuse, multiphase, and inhomogeneous, making it difficult to study \citep[e.g.,][]{tumlinson_circumgalactic_2017,mathur_probing_2022}; still, recent observations have confirmed the existence of metals at non-Solar abundance ratios in the CGM \citep[e.g.,][]{das_discovery_2019,das_hot_2021,gupta_supervirial_2021}. We investigate models where the infalling gas is pre-enriched to a metallicity
\begin{equation}
    \label{eq:pre-enrichment}
    Z_{\rm in}(t) = (1 - e^{-t/\tau_{\rm rise}}) Z_\odot 10^{\mathXH_{\rm CGM}}.
\end{equation}
In this case, the metallicity rises from 0 with a timescale $\tau_{\rm rise}=2\,{\rm Gyr}$ and plateaus at $\mathFeH_{\rm CGM}$. \added{We assume the Solar-scaled abundance} of the accreted gas is the same for all elements (i.e., $\mathXH_{\rm CGM}=\mathOH_{\rm CGM}$).

% \subsection{Infall Rate Parameter Selection}
% \label{sec:parameter-selection}

% \begin{figure*}
%     \centering
%     \includegraphics[width=\textwidth]{figures/infall_parameters.pdf}
%     \caption{Gas abundance tracks in the [O/Fe]--[Fe/H] plane for one-zone chemical evolution models with different infall history parameters. In each panel, one parameter is varied according to the legend while the other two are held fixed. The open symbols along each curve mark logarithmic steps in time, as denoted in panel (b). The marginal panels show the corresponding stellar abundance distributions, which are convolved with an 0.02 dex Gaussian kernel for visual clarity. All models use the same fiducial parameters as the $R_{\rm gal}=8\kpc$ ring in the multi-zone models, with \yZ{1} and $\eta=0.2$. The black lines in each panel represent the fiducial parameters.}
%     \label{fig:twoinfall-parameters}
% \end{figure*}

% Previous studies have adopted a wide range of parameter values for Equation \ref{eq:twoinfall-ifr}. Figure \ref{fig:twoinfall-parameters} illustrates the effect of varying the infall parameters on gas abundance tracks and stellar abundance distributions in a one-zone model with \yZ{1}. The first infall timescale $\tau_1$, shown in panel (a), primarily affects the stellar distribution along the high-$\alpha$ sequence. Though $\tau_1$ has an apparently large effect on the size of the low-$\alpha$ loop, the corresponding effect on the stellar abundance distribution is small due to the low number of stars formed between $t\sim3-6\,{\rm Gyr}$. It does, however, affect the position of the high-$\alpha$ peak, though this effect is more easily seen in the multi-zone model outputs. We adopt $\tau_1=1\,{\rm Gyr}$ for our fiducial value, similar to \citet{spitoni_galactic_2020} but longer than, e.g., \citet{nissen_high-precision_2020}, \citet{spitoni_apogee_2021} or \citet{hegedus_reconstructing_2025}, in order to set the peak of the high-$\alpha$ sequence at $\mathOFe\approx+0.35$ in the multi-zone models. 

% Panel (b) of Figure \ref{fig:twoinfall-parameters} shows that the second infall timescale $\tau_2$ controls the size of the low-$\alpha$ loop, which affects the width of the MDF and the low-$\alpha$ [O/Fe] distribution. A shorter $\tau_2$ produces a bigger loop and therefore a broader MDF and [O/Fe] distribution, while a longer $\tau_2$ produces a smaller loop, leading to both a narrower low-$\alpha$ sequence and a narrower MDF. We note that our maximum value of $\tau_2=30\,{\rm Gyr}$ is close to a constant infall rate, so a further increase in $\tau_2$ has diminishing effect. Between $\tau_2=3-30\,{\rm Gyr}$, the endpoint of the abundance tracks shifts by $\sim0.2$ dex in [Fe/H] and $\sim0.1$ dex in [O/Fe], which could affect the model's ability to reproduce the present-day abundance of the Solar neighborhood. We adopt a fiducial value of $\tau_2=15\,{\rm Gyr}$ for the Solar neighborhood in order to minimize the size of the loop and width of the low-$\alpha$ [O/Fe] distribution while still approaching Solar [Fe/H] at late times (see further discussion in Section \ref{sec:abundance-distributions}). This value is in line with the infall timescale recovered by \citet{spitoni_galactic_2020}, and similar to the local star formation timescale adopted by \citet{johnson_stellar_2021}, but significantly longer than the timescales found by \citet{nissen_high-precision_2020}, \citet{spitoni_apogee_2021}, and \citet{hegedus_reconstructing_2025}.

% In our multi-zone models, we vary the second infall timescale with radius to produce inside-out growth of the disk. Previous multi-zone two-infall studies \citep[e.g.,][]{chiappini_abundance_2001,palla_chemical_2020} scaled $\tau_2$ linearly with radius, with $\tau_2\approx1\Gyr$ in the inner disk and $\tau_2=7\Gyr$ at the Solar annulus. This prescription was adopted to match the metallicity distribution of the Solar neighborhood and the bulge in the absence of mass-loaded outflows \citep{romano_mass_2000}. We instead adopt an exponential $\tau_2 - R_{\rm gal}$ relation, with $\tau_2(R_\odot)=15\Gyr$ at the Solar annulus and a scale radius $R_{\tau_2}=7\kpc$. This is similar to the star formation history timescale of \citet{johnson_stellar_2021}, which was based on the  stellar age gradients in Milky Way-like spirals observed by \citet{sanchez_spatially_2020}. We have also run models with a linear prescription and with a uniform value for $\tau_2$ and have found little difference in our key results.

% Finally, panel (c) of Figure \ref{fig:twoinfall-parameters} shows that the time of maximum infall $t_{\rm max}$ strongly affects the overall stellar abundance distribution for values $t_{\rm max}\leq2\Gyr$, but in this case the gas tracks do not produce the characteristic abundance loop. For $t_{\rm max}>2\Gyr$, varying $t_{\rm max}$ results in a minor shift to the mean of the MDF and little change to the [O/Fe] distributions, even though the abundance tracks in [O/Fe]--[Fe/H] space appear very different. The value of $t_{\rm max}$ also slightly adjusts the ISM abundance endpoint, as a longer $t_{\rm max}$ means the chemical evolution ``reset'' from the second infall occurs closer to the present day (see discussion in Section \ref{sec:age-abundance}). We adopt a fiducial value of $t_{\rm max}=4.2\Gyr$, i.e. a lookback time of $9\Gyr$, which is generally in line with previous two-infall studies \citep[e.g.,][]{nissen_high-precision_2020,spitoni_galactic_2020,spitoni_apogee_2021,hegedus_reconstructing_2025}. This ensures that our models are compatible with the median age of the thick disk in the APOKASC-3 catalog of $9.14\pm0.05\Gyr$ \citep{pinsonneault_apokasc-3_2025}. 

% The Milky Way's last major merger with the dwarf galaxy, dubbed Gaia Sausage-Enceladus \citep[GSE;][]{belokurov_co-formation_2018,helmi_merger_2018}, has been proposed as an important influence on the transition from the thick disk to the thin disk, as in \citet{spitoni_remind_2024}. Our fiducial value of $t_{\rm max}=4.2\Gyr$ places the start of the formation of the thin disk close to the GSE merger (within uncertainties), which likely occurred $\sim10\Gyr$ ago \citep[e.g.,][]{helmi_merger_2018,gallart_uncovering_2019,naidu_reconstructing_2021,woody_rapid_2025}. However, recent arguments suggest that the GSE merger would not have led to two-infall-like evolution because the mass ratio is too low to produce sufficient dilution \citep{orkney_milky_2025}.

% We note that all our models are normalized to produce the same thick-to-thin-disk mass ratio of $f_{\Sigma}(R_\odot)=0.12$ \citep{bland-hawthorn_galaxy_2016} at the Solar annulus regardless of the infall parameters. The high-$\alpha$ sequence appears much less prominent in our [O/Fe] distributions in Figure \ref{fig:twoinfall-parameters} than in the data because the one-zone model outputs include only stars formed in-situ at the Solar annulus. In our multi-zone models, most of the high-$\alpha$ stars present in the Solar neighborhood have migrated from the inner Galaxy.

\subsection{Stellar Migration}
\label{sec:migration}

This study is not the first to apply a prescription for radial migration to a two-infall GCE model. \citet{spitoni_effect_2015} explored the effect of migration speeds of order $\sim1\,{\rm km}{\rm s}^{-1}$ on the metallicity distribution of the Solar neighborhood. 
% Assuming a constant migration speed, they prescribed some fraction of stars born in the inner and outer Galaxy that will contribute to the local present-day population, and they also assumed that some fraction of stars born in the Solar neighborhood will have migrated elsewhere by the present day. \citet{spitoni_effect_2015} improved agreement with the observed local metallicity distribution, but their method is difficult to apply to model predictions across the disk. 
\added{Our implementation,} based on the diffusion method of \citet{frankel_measuring_2018}, allows stellar populations to migrate between zones during each timestep, which allows us to explore the effects of radial migration over the entire disk. \citet{palla_mgfe_2022} compared the \citet{spitoni_effect_2015} prescription to that of \citet{frankel_measuring_2018} and found a similar effect on stellar abundance distributions.

The distance a stellar population born at $R_{\rm form}$ migrates over its age $\tau$ is drawn from a Gaussian centered at 0 with standard deviation
\begin{equation}
    \sigma_{\rm RM}(\tau,R_{\rm form}) = \sigma_{\rm RM8} \Big(\frac{\tau}{8\,{\rm Gyr}}\Big)^{0.33} \Big(\frac{R_{\rm form}}{8\kpc}\Big)^{0.61},
    \label{eq:radial-migration}
\end{equation}
where we adopt $\sigma_{\rm RM8}=2.68\kpc$ as the fiducial value for the strength of radial migration from \citet{dubay_galactic_2024}. This is smaller than the value of $\sigma_{\rm RM8}=3.6\kpc$ found by \citet{frankel_measuring_2018}, but in Section \ref{sec:age-abundance} we explore the effect of a stronger migration prescription.

All stellar populations are born at the Galactic midplane and are assigned a final midplane distance $z$ drawn from the distribution
\begin{equation}
    p(z|\tau,R_{\rm final}) = \frac{1}{4 h_z} {\rm sech}^2\Big(\frac{z}{2 h_z}\Big),
    \label{eq:sech-pdf}
\end{equation}
\added{\citep{spitzer_dynamics_1942}} where $R_{\rm final}$ is the final Galactocentric radius of the stellar population. The scale height $h_z$ is given by
\begin{equation}
    h_z(\tau,R_{\rm final}) = \Big(\frac{0.24\kpc}{e^2}\Big) \exp\Big(\frac{\tau}{7\,{\rm Gyr}} + \frac{R_{\rm final}}{6\kpc}\Big).
    \label{eq:scale-height}
\end{equation}
\added{The} final midplane distance is assigned at the end of the model run and therefore does not affect the chemical evolution. The parameters of Equations \ref{eq:radial-migration} and \ref{eq:scale-height} were chosen to fit the stellar migration patterns in the {\tt h277} hydrodynamical simulation \citep{christensen_implementing_2012}. A more complete discussion of the migration scheme and its consequences can be found in Appendix C of \citet{dubay_galactic_2024}.

\added{An} important distinction between our method and that of \citet{spitoni_effect_2015} \added{is that} SNe Ia from long-lived progenitors contribute Fe to each zone they migrate through, not just their birth zone. This is important because the median delay time of our SN Ia DTD is $\sim2$ Gyr, for which the width of the migration distribution is $\sigma_{\rm RM}\approx2$ kpc (Equation \ref{eq:radial-migration}). Therefore, a significant fraction of SN Ia progenitors born in a given zone will enrich a different region of the Galaxy.

\section{Multi-Zone Model Results}
\label{sec:multizone-results}

We run 11 multi-zone models at three different yield scales. For our \yZ{1} and \yZ{2} yield scales, we vary the SN Ia DTD, the migration strength $\sigma_{\rm RM8}$, the thick-to-thin disk mass ratio $f_\Sigma(R_\odot)$, and the enrichment of the infalling gas ${\rm [X/H]}_{\rm CGM}$. Our models and relevant parameters are summarized in Table \ref{tab:models}. In the remainder of this Section, we present results from these models and compare against APOGEE data and stellar age estimates.

\begin{deluxetable*}{lCCcDDD}
    \tablecaption{Summary of our multi-zone models and relevant parameters.\label{tab:models}}
    \tablehead{
        \colhead{Name} & \colhead{$y/Z_\odot$} & \colhead{$\eta_\odot$} & \colhead{SN Ia DTD} & \multicolumn2c{$\sigma_{\rm RM8}$ [kpc]} & \multicolumn2c{$f_\Sigma(R_\odot)$} & \multicolumn2c{${\rm [X/H]}_{\rm CGM}$}
    }
    \decimals
    \startdata
        {\tt yZ1-fiducial}  & 1 & 0.2   & Equation \ref{eq:plateau-dtd}  & 2.68  & 0.12  & $-\infty$   \\
        {\tt yZ1-migration} & 1 & 0.2   & Equation \ref{eq:plateau-dtd}  & 5.0   & 0.12  & $-\infty$   \\
        {\tt yZ1-diskratio} & 1 & 0.2   & Equation \ref{eq:plateau-dtd}  & 2.68  & 0.5   & $-\infty$   \\
        {\tt yZ1-preenrich} & 1 & 0.6   & Equation \ref{eq:plateau-dtd}  & 2.68  & 0.12  & -0.5      \\
        {\tt yZ1-best}      & 1 & 0.4   & Equation \ref{eq:plateau-dtd}  & 3.6   & 0.25  & -0.7      \\
        \hline
        {\tt yZ2-fiducial}  & 2 & 1.4   & Equation \ref{eq:plateau-dtd}  & 2.68  & 0.12  & $-\infty$   \\
        {\tt yZ2-powerlaw}  & 2 & 1.4   & Equation \ref{eq:powerlaw-dtd} & 2.68  & 0.12  & $-\infty$   \\
        {\tt yZ2-diskratio} & 2 & 1.4   & Equation \ref{eq:plateau-dtd}  & 2.68  & 0.5   & $-\infty$   \\
        {\tt yZ2-preenrich} & 2 & 2.4   & Equation \ref{eq:plateau-dtd}  & 2.68  & 0.12  & -0.5      \\
        {\tt yZ2-best}      & 2 & 1.8   & Equation \ref{eq:plateau-dtd}  & 3.6   & 0.25  & -0.7      \\
        \hline
        {\tt yZ3-fiducial}  & 3 & 2.4   & Equation \ref{eq:plateau-dtd}  & 2.68  & 0.12  & $-\infty$   \\
    \enddata
\end{deluxetable*}
\vspace{-24pt}

\subsection{Dilution \& Re-enrichment}
\label{sec:age-abundance}

\begin{figure}
    \centering
    \includegraphics[width=\onecolumn]{figures/gas_abundance_evolution.pdf}
    \caption{The ISM abundance evolution at $R_{\rm gal}=8\kpc$ of three multi-zone models at different yield scales (see Table \ref{tab:yields}), \added{each of which feature a major dilution event 9 Gyr ago. The dark gray (light gray) shaded region plots the 68\textsuperscript{th} (95\textsuperscript{th}) percentile of APOGEE abundances} in the Solar neighborhood ($7\leq R_{\rm gal}<9\kpc$, $|z|<0.5\kpc$) with NN ages from \citet{leung_variational_2023}. \added{The black line (black points) plots the median (mode)} of the abundance data in 1 Gyr-wide age bins, and the gray error bars along the bottom of each panel indicate the median age and abundance errors as a function of age. The left-hand marginal panels show the predicted (blue, pink, and green) and observed (gray) stellar abundance distributions, which are boxcar-smoothed with a width of 0.05 dex for visual clarity.
    % {\bf Key takeaway:} All models feature a major dilution event 9 Gyr ago, and the \yZ{1} model re-enriches the most slowly.
    }
    \label{fig:yield-outflow}
\end{figure}

% The two-infall scenario requires a large quantity of pristine or low-metallicity gas to be accreted onto the disk over a short span of time. As a natural consequence, the gas disk experiences a period of metal dilution, which should be recorded in the stellar record. 
\added{Figure \ref{fig:yield-outflow}} shows the mode of the APOGEE [O/H], [Fe/H], and [O/Fe] distributions \added{as a function of age at} the Solar annulus, using the NN ages from \citet{leung_variational_2023}. As shown by \citet{johnson_milky_2024}, the peak of the MDF is less susceptible to modification by radial migration, making it a more reliable proxy for ISM chemistry at a given lookback time than the mean or median.
% In agreement with recent work (see discussion in Section \ref{sec:introduction}), the trend with age is interestingly flat. [O/H] and [Fe/H] increase by $0.1-0.2\dex$, relative to young stars, in the $\sim6-8\Gyr$ age range, likely due to metal-rich stars migrating outward from the inner Galaxy where the surface densities are higher.
\added{We over-plot} the ISM abundance evolution in our $y/Z_\odot=1,2,$ and $3$ models with the fiducial parameters at $R_{\rm gal}=8\kpc$ \added{for comparison}. The models are in broad agreement with each other and with the data at lookback times of $\lesssim5\Gyr$. At ages of $\sim5-8\Gyr$, the models \added{underpredict} the metallicities of the observed stars by up to half an order of \added{magnitude, a natural} consequence of the dilution associated with the second infall event. This discrepancy is \added{modestly reduced in} the higher $y/Z_\odot$ models, which re-enrich more rapidly. \added{However}, these models exacerbate tensions with the age--[O/Fe] relation because the O abundance \added{increases more} before enrichment from second-infall SNe Ia becomes important, making the $\alpha$-enhancement from the ensuing starburst more pronounced. \added{Overall,} none of our models provide a good match to these age trends. Figure \ref{fig:yield-outflow} illustrates a point that will arise as a theme throughout the remainder of this paper, which is that the signatures of a substantial dilution event and subsequent re-enrichment, characteristic of the two-infall scenario, are simply not present in the observed age--abundance trends.

\begin{figure*}
    \centering
    \includegraphics[width=\linewidth]{figures/stellar_abundance_evolution.pdf}
    \caption{Stellar age--abundance relations predicted by multi-zone models at the \yZ{1} yield scale (see Table \ref{tab:yields}). Each point represents a stellar population drawn from the Solar neighborhood near the midplane ($7\leq R_{\rm gal}< 9\kpc$, $|z| < 0.5\kpc$) color-coded by its birth radius. \added{In this and subsequent figures, a} Gaussian scatter is applied to each point according to the median age and abundance uncertainties in Table \ref{tab:uncertainties}. For visual clarity, we plot only a random mass-weighted sample of \num{10000} points in each panel. The dashed curve plots the predicted ISM abundance at $R_{\rm gal}=8\kpc$, and the black squares plot the median stellar abundance in {2 Gyr}-wide age bins. \added{The red line and points are the median abundance for APOGEE stars, and the shaded regions are the 16th--84th percentile ranges. }
    % We adopt the NN age estimates for APOGEE stars from \citet{leung_variational_2023}.
    \added{Each column} shows results from a different multi-zone model: {\bf (a)} our fiducial model, {\tt yZ1-fiducial}; {\bf (b)} a model with stronger radial migration, {\yZ1-migration}; {\bf (c)} a model with a higher local thick-to-thin disk ratio, {\yZ1-diskratio}; and {\bf (d)} a model with pre-enriched gas infall, {\tt yZ1-preenrich} (see Table \ref{tab:models} for details). 
    % {\bf Key takeaway:} Models with a higher thick-to-thin disk ratio (c) or pre-enriched infall (d) reduce the magnitude of the ISM dilution, but neither can simultaneously match all of the observed age--abundance relation.
    }
    \label{fig:abundance-evolution-params}
\end{figure*}

Chemical evolution models that assume the \yZ{1} (empirical) yield scale already struggle to match the local age--metallicity relation even with a smooth SFH \citep[see also][]{johnson_milky_2024}. The problem is exacerbated in the two-infall case because of the delayed dilution \added{event---the approach} to equilibrium is ``reset'' by the second infall. The dilution of the ISM then gets baked into the stellar abundance record.
Figure \ref{fig:abundance-evolution-params} shows stellar age--abundance relations predicted by models with \yZ{1}. The model with the fiducial parameters (column (a); {\tt yZ1-fiducial}) shows similar discrepancies with the \citet{leung_variational_2023} NN age--[O/H] relation as in Figure \ref{fig:yield-outflow}. The evolution of [Fe/H] is similar, but the approach to the final metallicity is slower because of the additional delay imposed by SNe Ia. \added{We smoothed} simulated data points \added{based on} the median age and abundance errors in Table \ref{tab:uncertainties} \added{to incorporate these errors in the model}. The black dashed curve shows the evolution of gas phase abundances at $R_{\rm gal}=8\kpc$. Despite the effects of radial migration and statistical errors, the stellar populations mostly scatter around this gas phase curve, at least for $\tau\lesssim9\Gyr$.

We next attempt to mitigate the dilution and late-time evolution problems for the \yZ{1} yield scale. First, the observed rise in the median metallicity of stars in the $6-10\Gyr$ age range could be due to radial migration, as those stars were probably not born in-situ, but rather migrated from the metal-rich inner Galaxy \citep{feuillet_age-resolved_2018}. Column (b) of Figure \ref{fig:abundance-evolution-params} presents model {\tt yZ1-migration}, which has a stronger migration prescription of $\sigma_{\rm RM8}=5\kpc$. As a result, the stars that make up the present-day Solar neighborhood are drawn from a wider range of birth $R_{\rm gal}$, producing a broader abundance distribution at fixed age. Even though this prescription is extreme compared to the estimates of \citet{frankel_measuring_2018}, \added{for example}, the model still significantly under-predicts the metallicity of $\sim6-10\,{\rm Gyr}$ old stars.

Next, we explore model {\tt yZ1-diskratio}, which has a local thick-to-thin disk surface density ratio $f_\Sigma(R_\odot)=0.5$, $\sim4$ times larger that the fiducial value. This is higher than most of the constraints from population counts or GCE models (see Section \ref{sec:sfh}). Column (c) of Figure \ref{fig:abundance-evolution-params} shows that requiring a more massive thick disk can reduce the dilution and recent evolution of the ISM because more of the gas disk is built up during the first infall phase. Model {\tt yZ1-diskratio} produces the best agreement \added{of the four models in Figure \ref{fig:abundance-evolution-params} to} the observed age--[Fe/H] relation (second row). However, agreement with the observed age--[O/Fe] relation is poor. The model predicts a much flatter trend than observed, under-predicting [O/Fe] by $\sim0.1$ dex in the $\sim6-10\Gyr$ age range.

Finally, we investigate model {\tt yZ1-preenrich}, where the infalling gas is enriched to a metallicity $\mathOH=\mathFeH={\rm [X/H]}_{\rm CGM}$ before accreting onto the disk. Column (d) of Figure \ref{fig:abundance-evolution-params} shows results for the case where ${\rm [X/H]}_{\rm CGM}=-0.5$, the highest metallicity allowed by the local low-$\alpha$ population. 
% Pre-enriched infall also raises the equilibrium metallicity, so we increase the mass-loading factor to $\eta=0.6$ to maintain the end-point at roughly Solar metallicity. 
\added{Pre-enriched infall} at this level mitigates but does not completely solve the two discrepancies. The dilution effect of the second infall is reduced to the $\sim0.3$-dex level as the gas which replenishes the Galaxy's reservoir is no longer pristine; however, the width of the stellar abundance distribution at any given age is also reduced, since the accreting gas is less chemically different from the ISM, diminishing the effects of dilution. This model also narrows the [O/Fe] distribution of mono-age populations (almost all the model stars fall within the $1\sigma$ band of the data), which could be compensated for by stronger radial migration.

Overall, no modification to the \yZ{1} model is able to completely overcome the issues that dilution and re-enrichment naturally face when confronted with a flat age--metallicity relation. Pre-enrichment of the accreted gas and a higher disk mass ratio can reduce the discrepancy with the data, but these options introduce new issues in the age--[O/Fe] plane. 

\subsection{Abundance Evolution Across the Disk}
\label{sec:disk-evolution}

\begin{figure*}
    \centering
    \includegraphics[width=\textwidth]{figures/mdf_evolution.pdf}
    \caption{\added{Decomposition of the present-day MDF by stellar age across the Galactic disk. 
    % In each panel, normalized MDFs within a {2 kpc}-wide annulus are color-coded by the stellar age range. 
    The gray curve plots the total present-day MDF in each region. The distributions in all panels are restricted to $|z|<0.5\kpc$ and boxcar-smoothed with a width of {0.1 dex} for visual clarity. Rows (a) and (b) present the distributions from the {\tt yZ1-fiducial} and {\tt yZ2-fiducial} models, respectively} (see Table \ref{tab:models}). 
    % A Gaussian scatter has been applied to each model stellar population in rows (a) and (b) according to the median [C/N]-derived age and abundance uncertainties in Table \ref{tab:uncertainties}. 
    Row (c) presents the distributions from APOGEE with ages derived from [C/N] abundances by \citet{roberts_cn_2025} (see Section \ref{sec:age-estimates}). \added{In row (c), the vertical blue dotted lines mark the mode of the distribution of the $1-2\kpc$ age bin for reference, and the gray dashed line marks the cut at $\mathFeH\ge-0.4$ for upper RGB and RC stars.}%, and the gray solid line marks the cut at $\mathFeH<+0.45$ for all stars with [C/N]-based ages.
    % {\bf Key takeaway:} The APOGEE distributions show remarkably little variation in mode [Fe/H] over the past $\sim6-8$ Gyr at all radii, whereas both two-infall models predict a steady evolution toward higher [Fe/H] with time.
    }
    \label{fig:mdf-evolution}
\end{figure*}

The discrepancies between the predicted and observed abundance evolution in the Solar neighborhood discussed in Section \ref{sec:age-abundance} persist across the Galactic disk. Figure \ref{fig:mdf-evolution} \added{breaks down the MDF by stellar age} across five radial bins for the {\tt yZ1-fiducial} and {\tt yZ2-fiducial} models. For the APOGEE sample, we use the [C/N]-derived age estimates due to the larger sample size in the most distant regions of the disk; we limit the comparison to ages in the range $1-10$ Gyr because the [C/N] ages are most reliable in this range, as discussed in \added{Section \ref{sec:age-estimates}.} 

The predictions of both models in Figure \ref{fig:mdf-evolution} show a clear trend in [Fe/H] with age at all radii. The MDF shifts consistently toward high metallicity when moving from older to younger stars. The distance between the $1-2\Gyr$ and $2-4\Gyr$ age bins is smaller for the \yZ{2} model because of the faster approach to equilibrium (see also Figure \ref{fig:yield-outflow}). In the Solar annulus (center column), the peak of \added{the MDF of $6-8\Gyr$ old stars} is $0.3\dex$ lower in the \yZ{2} model than observed, and $0.4\dex$ lower in the \yZ{1} model.

In contrast, the APOGEE data show remarkably little evolution in [Fe/H] up to ages of $\sim8\Gyr$ at all radii. Row (c) of Figure \ref{fig:mdf-evolution} shows that the MDF broadens with age, but its peak remains constant across this range. The mode [Fe/H] for the youngest stars (indicated by the vertical blue dotted line) is nearly the same as for the $6-8\Gyr$ old stars. At $R_{\rm gal}<7\kpc$, the MDF skews to lower [Fe/H] more noticeably with age, but its mode does not shift by more than $\sim0.1\dex$. It is difficult to draw conclusions about the outer Galaxy because the mode [Fe/H] is close to the metallicity cut at $\mathFeH>-0.4$ for luminous giants (represented by the vertical gray dashed line), which comprise the majority of stars in the sample at that distance. The remarkable consistency of the MDF over time, the result that motivated the equilibrium scenario proposed by \citet{johnson_milky_2024}, contrasts with the predictions of our fiducial models.

The oldest age bin in Figure \ref{fig:mdf-evolution} shows distinct behavior in both the models and data. The $8-10\Gyr$ age bin spans both the tail end of the thick disk phase and the beginning of the thin disk, so the MDF is much broader than in other age bins. In the models, the main, metal-poor peak consists of $8-9\Gyr$ old stars (post-dilution), and the extended high-[Fe/H] tail consists of $>9\Gyr$ old stars (pre-dilution). Intriguingly, the APOGEE MDF in that age bin is also very broad and is bimodal in \added{the two inner-most radial bins}, peaking at $\mathFeH\approx-0.3$ and $+0.3$. \added{While} the data and the models show qualitatively similar behavior, they actually represent different populations. In the data, the metal-rich peak are all low-$\alpha$ stars, while the metal-poor peak is the locus of the high-$\alpha$ sequence---a reversal of the model predictions.

\subsection{The [O/Fe] Distribution}
\label{sec:abundance-distributions}

\begin{figure}
    \centering
    \includegraphics[width=\onecolumn]{figures/ofe_feh_density.pdf}
    \caption{The density of stars in the [O/Fe]--[Fe/H] plane predicted by multi-zone models with (a) $y/Z_\odot=1$ and (b) $y/Z_\odot=2$, and (c) from the APOGEE DR17 catalog. The curves in panels (a) and (b) plot the ISM abundance at the Solar annulus over time, and the alternating black and white segments mark time intervals of {1 Gyr}. The model output has been re-sampled to match the APOGEE stellar $|z|$ distribution.
    % and a Gaussian scatter has been applied to the predicted abundances according to Table \ref{tab:uncertainties}. 
    \added{Stars} in all panels are restricted to the region defined by $7\leq R_{\rm gal}< 9\kpc$ and $|z|<2\kpc$. 
    % {\bf Key takeaway:} both models predict a stellar over-density at intermediate [O/Fe] and low metallicity, which is not observed in APOGEE.
    }
    \label{fig:ofe-feh-density}
\end{figure}

In the two-infall scenario, the low-$\alpha$ loop governs the chemical evolution of the thin disk \added{\citep[see Section 6 of][]{spitoni_galactic_2019}}. However, careful inspection of the marginal [O/Fe] distributions in \added{Figure \ref{fig:yield-outflow}} reveals a different morphology: the models predict {\it three} peaks in the [O/Fe] distribution, whereas the data show only two. The location of the intermediate peak varies depending on the yields and model parameters, but is always present. This morphology remains essentially consistent in our multi-zone models as well, despite the inclusion of radial mixing and vertical dispersion of stars.

Figure \ref{fig:ofe-feh-density} illustrates the origin of the intermediate-$\alpha$ peak predicted by the two-infall model. Both the models with $y/Z_\odot=1$ and $y/Z_\odot=2$ predict an over-density of stars near the abundance turn-over ($\mathFeH\approx-0.3$, $\mathOFe\approx0.1-0.2$), which is not seen in the APOGEE sample. The local maximum in $dN_\star/d\mathOFe$ comprises stars which formed in the second infall before SN Ia enrichment began to drive down [O/Fe]. This prediction should therefore arise in {\it any} two-infall model regardless of its specific parameters, but its impact can be mitigated through parameter choices that act to compress the distance between the low- and intermediate-$\alpha$ peaks, as in the $y/Z_\odot=1$ model in Figure \ref{fig:yield-outflow}.

Additionally, the shape of the low-$\alpha$ sequence predicted by the models (a concave-down ``comma'') is different from the data (a concave-up ``swoosh''). This is a subtle distinction, on the $\sim0.05\dex$ level in [O/Fe]. We have not accounted for the APOGEE selection function in our models, but there is no evident reason for observational effects to produce this difference in morphology. This problem is not unique to the two-infall scenario: it results from the evolution toward the lower right of the [O/Fe]--[Fe/H] plane and \added{is typical of other models} \citep[e.g.,][]{minchev_chemodynamical_2013,johnson_stellar_2021,prantzos_origin_2023}. The two-infall scenario is otherwise fairly successful at reproducing the local stellar distribution in [O/Fe]--[Fe/H] space.

\begin{figure*}
    \centering
    \includegraphics[width=\textwidth]{figures/ofe_distributions.pdf}
    \caption{Normalized stellar [O/Fe] distributions predicted by multi-zone models with \yZ{2} (a--d) and as observed by APOGEE (e). Each row presents stellar distributions within a range of absolute midplane distance $|z|$ reported on the far right, and the vertical scale is consistent across each row. Within each panel, the distributions are color-coded according to the bin in galactocentric radius $R_{\rm gal}$ from which they are drawn. 
    % The median APOGEE abundance uncertainties are forward-modeled onto the model outputs (see Table \ref{tab:uncertainties}). 
    \added{For visual clarity}, each distribution is smoothed with a box-car of width 0.05 dex.
    Each column shows the distributions predicted from a different multi-zone model: {\bf (a)} the fiducial model with \yZ{2}, {\tt yZ2-fiducial}; {\bf (b)} a model with a power-law DTD, {\tt yZ2-powerlaw}; {\bf (c)} a model with a higher local thick-to-thin disk ratio, {\tt yZ2-diskratio}; and {\bf (d)} a model with pre-enriched gas infall, {\tt yZ2-preenrich} (see Table \ref{tab:models} for details). 
    % {\bf Key takeaway:} For the \yZ{2} case, a higher thick-to-thin disk ratio or pre-enrichment of the accreted gas can improve the low-$\alpha$ distribution while preserving the high-$\alpha$ peak.
    }
    \label{fig:ofe-distribution-parameters}
\end{figure*}

Next, we attempt to mitigate the intermediate peak through several parameter choices. Figure \ref{fig:ofe-distribution-parameters} compares [O/Fe] distributions from across the Galactic disk predicted by models with \yZ{2}. We present the distributions in multiple bins of $|z|$ as well as $R_{\rm gal}$ because the observed pattern varies as a function of midplane distance, and because the APOGEE selection function over-emphasizes high-$|z|$, and therefore high-$\alpha$, stars in the full sample \citep[see Figure 5 from][]{vincenzo_distribution_2021}. The {\tt yZ2-fiducial} model (column (a)) predicts a high density of stars at $\mathOFe\approx+0.2$, where the data instead show a trough. The peak is most pronounced in the inner disk ($R_{\rm gal}=3-5\kpc$) because of the shorter infall timescale.

First, we substitute our fiducial SN Ia DTD with a simple power-law (Equation \ref{eq:powerlaw-dtd}; {\tt yZ2-powerlaw}),
which reduces the median SN Ia delay time from $\sim2\,{\rm Gyr}$ to $\sim0.5\,{\rm Gyr}$. As shown in column (b), this has the intended effect on the low-$\alpha$ sequence, but it also entirely eliminates the high-$\alpha$ peak. \citet{dubay_galactic_2024} discuss in detail why such a DTD is disfavored by Milky Way stellar abundances for several different SFHs including for the two-infall scenario.

Next, in model {\tt yZ2-diskratio} (column (c)) we increase the local thick-to-thin disk surface density ratio by a factor of 4 to $f_\Sigma(R_\odot)=0.5$. This value means that 1 in 3 stars in the Solar annulus belong to the thick disk, which is on the high end of estimates (see Section \ref{sec:sfh}). The result as shown in Figure \ref{fig:ofe-distribution-parameters} is a true bimodal abundance distribution, with a more prominent high-$\alpha$ peak than in the other models.

Finally, in model {\tt yZ2-preenrich} (column (d)) the metallicity of the infalling gas increases to $\mathXH_{\rm CGM}=-0.5$ at late times. This model results in very similar [O/Fe] distributions to the $y/Z_\odot=1$ case. We assume that the infalling gas has $\mathOFe_{\rm CGM}=0$ at all times; an alternate model with $\mathOFe_{\rm CGM}=+0.3$ shifted the distribution towards higher [O/Fe], worsening agreement with observations. In summary, either an enhanced disk mass ratio or pre-enriched infall can improve agreement with the observed thin disk abundances for the \yZ{2} case. These parameters also help the model better fit the age--metallicity relation, as shown in Section \ref{sec:age-abundance} for the \yZ{1} case.

\subsection{The Best Model}
\label{sec:ofe-feh-best}

\begin{figure*}
    \centering
    \includegraphics[width=\linewidth]{figures/ofe_feh_best.pdf}
    \caption{Stellar abundance distributions across the disk predicted by our best multi-zone model, with \yZ{2}, ${\rm [X/H]}_{\rm CGM}=-0.7$, $f_\Sigma(R_\odot)=0.25$, $\sigma_{\rm RM8}=3.6\kpc$, and $\eta_\odot=1.8$. Each panel presents a random mass-weighted sample of \num{10000} stellar populations color-coded by age. 
    % A Gaussian scatter is applied to each point according to the median [C/N]-derived age and abundance uncertainties in Table \ref{tab:uncertainties}. 
    \added{The solid (dashed) contours enclose 30\% (80\%) of the APOGEE stars in each region.} 
    % {\bf Key takeaway:} The predicted distribution from the two-infall model lines up with the APOGEE distribution close to the midplane, but agreement is worse at higher latitudes and in the outer Galaxy.
    }
    \label{fig:ofe-feh-best}
\end{figure*}

Motivated by the results of the previous sections, we attempt to construct a model that solves all of the issues that have been outlined thus far. Our ``best attempt'' model {\tt yZ2-best} uses the \yZ{2} yield set to flatten the local age--metallicity relation (Figure \ref{fig:yield-outflow}), pre-enriched infall at the level of $\mathXH_{\rm CGM}=-0.7$ to reduce the dilution at $t_{\rm max}$ (Figure \ref{fig:abundance-evolution-params}), slightly stronger outflows with $\eta_\odot=1.8$ to maintain the local equilibrium at Solar metallicity, moderately stronger radial migration with $\sigma_{\rm RM8}=3.6\kpc$ to widen the local metallicity dispersion (Figure \ref{fig:abundance-evolution-params}), and a greater local disk ratio $f_\Sigma(R_\odot)=0.25$ to reduce the width of the low-$\alpha$ distribution and beef up the high-$\alpha$ sequence (Figure \ref{fig:ofe-distribution-parameters}). Our choices for $\mathXH_{\rm CGM}$, $\sigma_{\rm RM8}$, and $f_\Sigma(R_\odot)$ are more moderate \added{than the illustrative examples in the previous section. Our} focus is on qualitative rather than quantitative agreement with the data, and thus we do not attempt to find the optimal set of parameters.

Figure \ref{fig:ofe-feh-best} shows the stellar [O/Fe]--[Fe/H] distributions in different ranges of $R_{\rm gal}$ and $|z|$ color-coded by age as predicted by the {\tt yZ2-best} model. This model is generally successful at reproducing the observed abundance distributions, especially in the inner Galaxy and close to the midplane. However, the predicted high-$\alpha$ sequence has a significant presence even in the outer Galaxy---likely a consequence of the stronger migration prescription and higher thick-to-thin disk ratio. In general, the predicted distributions do not align with the data as well at large midplane distances ($1\leq|z|<2\kpc$), but this may reflect inaccuracies in our prescription for vertical heating (see Section \ref{sec:migration}).

The {\tt yZ2-best} model makes two notable predictions about the age--abundance distributions. First, there is a population of $\sim8-9\Gyr$ old stars at sub-Solar [O/Fe], especially at $|z|\ge0.5\kpc$, formed immediately after the second infall during a period of rapid chemical evolution. These stars form a small percentage of the overall distribution \citep[see also Figure 11 from][]{spitoni_remind_2024} but they occupy a unique portion of the abundance space. 
% A longer $\tau_1$ or smaller $t_{\rm max}$ could shift this population to higher [O/Fe], where it would be obscured by the rest of the low-$\alpha$ sequence (see Figure \ref{fig:twoinfall-parameters}). 
\added{Second}, the stars born at the tail end of the thick and thin disk epochs are adjacent to each other in abundance space, meaning the two-infall scenario predicts a bimodal age distribution for metal-rich stars. The following section explores this second prediction in greater detail.

\subsection{Local Age Patterns}

\begin{figure*}
    \centering
    \includegraphics[width=\textwidth]{figures/lmr_ages.pdf}
    \caption{{\it Top:} The median stellar age as a function of [O/Fe] and [Fe/H] in the Solar annulus ($7\leq R_{\rm gal}<9\kpc$, $0\leq|z|<2\kpc$). The left and center panels \added{show} the {\tt yZ1-best} and {\tt yZ2-best} models, with ${\rm [X/H]}_{\rm CGM}=-0.7$, $f_\Sigma(R_\odot)=0.25$, and $\sigma_{\rm RM8}=3.6\kpc$ (see Table \ref{tab:models}). The model output has been re-sampled to match the APOGEE stellar $|z|$ distribution, and a Gaussian scatter has been applied to the abundances and ages according to Table \ref{tab:uncertainties}. The right panel plots the results from APOGEE using the \citet{leung_variational_2023} NN age catalog. The contours \added{show} the density of stars in the [Fe/H]--[O/Fe] plane, and the vertical dashed line denotes the boundary for locally metal-rich (LMR) stars.
    {\it Bottom}: Stellar age distributions in the Solar annulus for all stars (black) and LMR stars (gray). The left and center panels plot the mass-weighted age distributions predicted by the models,
    % after forward-modeling age uncertainties, 
    \added{and the right panel plots the NN ages for APOGEE stars.}
    % {\bf Key takeaway:} Both two-infall models predict a fundamentally different age pattern than what is observed, especially for LMR stars.
    }
    \label{fig:lmr-ages}
\end{figure*}

The two-infall scenario makes a fundamental prediction about the local stellar age distribution: the most metal-rich stars born in-situ in any region of the Galaxy should come from the metal-rich tail of the first infall sequence, and should therefore be older than all of the thin disk stars. As noted in the previous section, this prediction is apparent in any of the panels in Figure \ref{fig:ofe-feh-best}, especially where $|z|<0.5\kpc$. The top row of panels in Figure \ref{fig:lmr-ages} presents the median stellar age as a function of [O/Fe] and [Fe/H] predicted by the {\tt yZ1-best} and {\tt yZ2-best} models (see Table \ref{tab:models}) and observed in APOGEE using the \citet{leung_variational_2023} NN ages. While the models predict a fairly accurate distribution of stars in abundance space, especially for the low-$\alpha$ population, the stellar age patterns are starkly different. In both models, there is a sharp divide in the median stellar age when moving from the thick disk ($\tau\ge9\Gyr$) to the thin disk ($\tau\lesssim5\Gyr$). The \yZ{2} model also predicts that the stars with the lowest [O/Fe] should be $\sim8-9\Gyr$ old, while these are some of the youngest stars in APOGEE. The latter issue can be mitigated by adjusting the parameters of the first infall, as discussed in the previous Section, but the former is not solved so easily.

We further highlight the discrepant age patterns in the bottom panels of Figure \ref{fig:lmr-ages}, which compare the overall stellar age distribution against that of the locally metal rich (LMR) stars, defined here as $\mathFeH\ge+0.1$.\footnote{
    While the amplitude of the peaks is sensitive to the precise location of the LMR cutoff, the dearth of intermediate-age stars is robust to adjustments of $\pm0.05\dex$.
} For APOGEE, the distributions of all stars and only LMR stars are similar, both peaking near $\sim5\Gyr$, although very few of the LMR stars have ages $\gtrsim10\Gyr$. Our two-infall models predict an overall age distribution that is similar to the data, but both models predict a distinctly bimodal age distribution for LMR stars. The two populations reflect the metal-rich endpoints of the two successive infall epochs. The trough between the modes lies at $\sim5\Gyr$ for both models, right where the APOGEE distribution peaks. It is difficult to \added{obtain} a unimodal age distribution with the available parameter space. A dearth of intermediate-age, high-metallicity stars is a natural outcome of a substantial dilution event, which is a defining feature of the two-infall scenario. None of the adjustments that we explored in Sections \ref{sec:age-abundance} or \ref{sec:abundance-distributions} substantially increase the proportion of intermediate-age stars.

\section{Discussion}
\label{sec:discussion}

In this work, we have highlighted a few predictions of the two-infall scenario and compared against trends from large stellar age catalogs. We have explored a wide range of parameter space, but our investigation is not intended to be exhaustive. In this section, we discuss other parameters explored by previous studies as well as further extensions to the two-infall model.

\subsection{Third Accretion Episode}

Motivated by evidence of a recent period of enhanced star formation across the MW disk \citep{ruiz-lara_recurrent_2020}, \citet{spitoni_beyond_2023} extended the two-infall scenario with a recent $(\lesssim3\Gyr)$ third accretion episode. In contrast to the two-infall model of \citet{spitoni_apogee_2021}, which predicted a present-day gas metallicity of ${\rm [M/H]}\approx+0.3$ in the Solar neighborhood, \citet{spitoni_beyond_2023} argued that the gas dilution resulting from the third infall could explain a population of young stars with slightly sub-Solar abundances discovered in {\it Gaia} DR3 \citep{recio-blanco_gaia_2023}.

Similarly, \citet{palla_mapping_2024} invoked a late-time accretion episode to explain largely age-independent metallicites in open clusters across the disk. Open clusters trace younger populations, and most of their clusters are $\lesssim2\Gyr$ old. A third infall can help explain trends within this limited age range, but when compared to the full range of stellar age data, it faces the same challenges as the second infall. Successive epochs of dilution and re-enrichment will always struggle to explain chemical abundances that remain nearly constant with age over a $\sim10\Gyr$ period. As discussed in Section \ref{sec:age-abundance}, pre-enriched infall could reduce the magnitude of the dilution from a recent accretion episode, but the stellar age data place tight constraints on the parameter space.

\subsection{Radial Gas Flows}
\label{sec:radial-flows}

Radial gas flows are a potential alternative to the outflow prescription that we use in this paper. Within the two-infall paradigm, \citet{spitoni_effects_2011} found that inward flow velocities of a $\sim$ few $\kms$ can steepen the predicted radial metallicity gradient to match observations. Similarly, \citet{palla_chemical_2020,palla_mapping_2024} used an inward flow with a speed of $\sim1\kms$ to reproduce the observed radial metallicity profile in the absence of Galactic outflows. \added{\citet{johnson_constraints_2025}} considered several possible scenarios for the processes that drive radial gas flows. Their findings support these previous arguments that radial gas flows \added{can help to regulate the overall metal abundance and radial gradient.}

In our models, the outflows regulate the overall metallicity in a radially dependent manner, which \citet{johnson_milky_2024} demonstrated can improve agreement with age trends across the Galactic disk. 
% Johnson et al. (2025, in preparation) found similar results for radial gas flows with velocities that are relatively constant in both Galactocentric radius and time. At least one radial gas flow scenario therefore has similar effects on metal enrichment as our outflow prescription. However, 
\added{The outflow is simpler in implementation than a radial gas flow}; we can simply adjust the mass loading factors up or down in combination with a given scale of yields. This approach allows us to easily consider models that predict metallicity to evolve on different timescales but reach similar abundances at the present day (see, e.g., Figure \ref{fig:yield-outflow}). 
% A radial gas flow requires at least one assumption about the processes that drive the flow and generally does not afford the same level of flexibility. 
\added{We have therefore focused on models with outflows in this paper for the sake of simplicity, but constant} radial gas flow models should have similar effects.

\subsection{Star Formation Hiatus}
\label{sec:sfe-hiatus}

\begin{figure}
    \centering
    \includegraphics[width=\onecolumn]{figures/sfe_hiatus.pdf}
    \caption{Abundance tracks and distributions from one-zone models which experience an efficiency-driven starburst. The blue dashed curve represents the fiducial model that has an exponentially declining infall rate with timescale $\tau_{\rm in}=15\Gyr$, and constant star formation efficiency timescale $\tau_\star=2\,{\rm Gyr}$. The red solid curve plots the output of a model which \added{has an enhancement of $\tau_\star$ by a factor of 10 for 200 Myr} starting at $t=1.4\,{\rm Gyr}$. Both models assume the \yZ{2} yield set, with $y_{\rm Fe}^{\rm Ia}$ reduced by 20\% to better match the model endpoint with the data, and $\eta=1.4$. The grayscale histogram \added{is} the number density of APOGEE stars in the Solar annulus ($7\leq R_{\rm gal}< 9\kpc$, $|z|<2\kpc$) in [O/Fe]--[Fe/H] space, and the gray histograms in the marginal panels show the APOGEE stellar abundance distributions.}
    \label{fig:onezone-sfe-hiatus}
\end{figure}

The two-infall scenario falls into the broader category of GCE models that reproduce the $\alpha$-bimodality by halting or severely limiting star formation \added{temporarily}. For the two-infall model, this phase of low star formation lasts for a few Gyr \added{between} the $e$-folding timescale of the first infall and the onset of the second. However, as we have shown in Section \ref{sec:age-abundance}, the resulting dilution of the ISM is in tension with the observed stellar age--metallicity relation.

An accretion-driven star formation history is not the only way to induce a hiatus in star formation. Investigating a simulated galaxy from the Illustris TNG50 suite \citep{pillepich_first_2019,nelson_first_2019,nelson_illustristng_2019}, \citet{beane_rising_2024} argued that its MW-like $\alpha$-bimodality is the result of a brief ($\sim300\,{\rm Myr}$) quiescent period caused by the formation of a bar. In this scenario, the newly-formed bar funnels gas into the galaxy's central black hole, prompting a burst of AGN activity which suppresses star formation. Unlike our two-infall model, where the gas mass suddenly grows by a factor of $\sim5$ during the 1 Gyr following the second infall, the mass of their galaxy grows slowly during the quiescent period, indicating a dearth of major accretion events.

While our semi-analytic model does not include a Galactic bar, we can induce a star formation hiatus by artificially boosting the SFE timescale $\tau_\star$ for a period of time. Figure \ref{fig:onezone-sfe-hiatus} illustrates the effect of this efficiency-driven hiatus in a one-zone GCE model with an exponentially declining infall rate ($\tau_{\rm in}=15\Gyr$). During the quiescent period, the [O/Fe] ratio slowly declines due to the delayed contribution of Fe from SNe Ia. Meanwhile, the gas mass continues to increase through accretion even as star formation is suppressed. When $\tau_\star$ is lowered at the end of the quiescent period, it sparks a moderate star formation burst which causes stellar abundances to ``pile up'' at similar [O/Fe] values. The trough between the high- and low-$\alpha$ sequences results from the star formation returning to pre-quiescence behavior. The onset time of the SFE hiatus controls the position of the high-$\alpha$ sequence\textemdash a later onset places the peak at lower [O/Fe]\textemdash while the duration and enhancement factor control the size of the high-$\alpha$ peak.

The parameters of the SFE hiatus in Figure \ref{fig:onezone-sfe-hiatus} were chosen to match the APOGEE stellar [O/Fe] distribution as closely as possible. As such, we start the hiatus at $t=1.4\Gyr$ to match the high-$\alpha$ locus, even though this is $\sim3\Gyr$ before the median age of the thick disk \added{measured by} \citet{pinsonneault_apokasc-3_2025}. This hiatus model illustrates an alternative scenario for reproducing the $\alpha$-bimodality, but a full multi-zone treatment is beyond the scope of this work.

\section{Summary \& Conclusions}
\label{sec:conclusions}

We have compared the predictions of the two-infall scenario against abundance data from APOGEE DR17 supplemented with age estimates using two different methods. We ran multi-zone GCE models at two different yield scales with prescriptions for radially-dependent outflows and stellar migration. While the two-infall scenario can explain the distinct high- and low-$\alpha$ sequences, it faces challenges in matching the age--abundance structure of the full disk. We explored multiple parameter modifications to bring the model predictions closer to the data, including the yield scale, radial migration strength, metallicity of the accreted gas, thick-to-thin disk mass ratio, and the SN Ia DTD. We found that our two-infall models faced the following \added{challenges}:

\begin{itemize}
    \item By the model's design, the Galaxy undergoes a period of \added{rapid} dilution and subsequent re-enrichment. Models at a lower yield scale and weaker outflows predict slower re-enrichment. By contrast, the observed stellar age--metallicity relation is roughly constant over the past $\sim10\Gyr$. Pre-enriching the accreted gas can reduce the magnitude of the dilution, but \added{cannot} eliminate it entirely.
    \item Due to this re-enrichment, our models predict that the MDF shifts to higher metallicity with decreasing age throughout the disk. This contrasts with the APOGEE data, which show very little change in the mode of the MDF over the past $\sim6-8$ Gyr.
    \item For metal-rich stars, the two-infall model predicts a sharp divide in the stellar age distribution between the thick and thin disk populations: local metal-rich stars should either be very old or very young. In contrast, the data show a broad, unimodal age distribution, with most of the metal-rich stars having intermediate ages. This discrepancy could not be reconciled using any of the model variations we explored.
    \item The ``turn-over'' in the evolution of [O/Fe] following the second infall produces a double-peaked low-$\alpha$ sequence with a fundamentally different abundance structure than observed, especially for models with higher yields. A low yield set (\yZ{1}) coupled with lower outflows, or pre-enrichment of the infalling gas, can bring the stellar [O/Fe] distributions more in line with the data. The parameter space is nonetheless restricted by the need to suppress this feature.
\end{itemize}

\added{We} have not found a GCE model, two-infall or otherwise, that is capable of reproducing all of the \added{observations}. The ``inside-out'' models of \citet{johnson_stellar_2021} are better able to predict a slowly evolving metallicity gradient, but they do not predict an $\alpha$-bimodality. Even a smoothly-evolving SFH struggles to reproduce the flat age--metallicity relation at the lower empirical yield scale \citep{johnson_milky_2024}. We emphasize, though, that by its very construction, the two-infall scenario trades success at reproducing certain observations for failure at others.

The apparent age-independence of stellar abundances in the disk places considerable restrictions upon the two-infall parameter space because it predicts a substantial dilution event at the start of the thin disk epoch. If the equilibrium scenario of \citet{johnson_milky_2024} is accurate, then it restricts the two-infall scenario more than other GCE models. Restrictions on the two-infall scenario also translate more broadly to other merger-dominated SFHs, which must explain the MW's $\alpha$-bimodality without significantly diluting the metallicity of the ISM.

\section*{Acknowledgements}

We are grateful to Prof.\ David Weinberg for his thoughtful comments and input on this paper, \added{and to Prof.\ Christopher Kochanek for his useful comments on the manuscript}. We also thank Dr.\ Emanuele Spitoni for useful conversations about this work.

LOD and JAJ acknowledge support from National Science Foundation grant no.\ AST-2307621. JAJ and JWJ acknowledge support from National Science Foundation grant no.\ AST-1909841.
LOD acknowledges financial support from an Ohio State University Fellowship.
JWJ acknowledges financial support from a Carnegie Theoretical Astrophysics Center postdoctoral fellowship.

Funding for the Sloan Digital Sky 
Survey IV has been provided by the 
Alfred P.\ Sloan Foundation, the U.S.\ 
Department of Energy Office of 
Science, and the Participating 
Institutions. 

SDSS-IV acknowledges support and 
resources from the Center for High 
Performance Computing  at the 
University of Utah. The SDSS 
website is \url{www.sdss4.org}.

SDSS-IV is managed by the 
Astrophysical Research Consortium 
for the Participating Institutions 
of the SDSS Collaboration including 
the Brazilian Participation Group, 
the Carnegie Institution for Science, 
Carnegie Mellon University, Center for 
Astrophysics | Harvard \& 
Smithsonian, the Chilean Participation 
Group, the French Participation Group, 
Instituto de Astrof\'isica de 
Canarias, The Johns Hopkins 
University, Kavli Institute for the 
Physics and Mathematics of the 
Universe (IPMU) / University of 
Tokyo, the Korean Participation Group, 
Lawrence Berkeley National Laboratory, 
Leibniz Institut f\"ur Astrophysik 
Potsdam (AIP),  Max-Planck-Institut 
f\"ur Astronomie (MPIA Heidelberg), 
Max-Planck-Institut f\"ur 
Astrophysik (MPA Garching), 
Max-Planck-Institut f\"ur 
Extraterrestrische Physik (MPE), 
National Astronomical Observatories of 
China, New Mexico State University, 
New York University, University of 
Notre Dame, Observat\'ario 
Nacional / MCTI, The Ohio State 
University, Pennsylvania State 
University, Shanghai 
Astronomical Observatory, United 
Kingdom Participation Group, 
Universidad Nacional Aut\'onoma 
de M\'exico, University of Arizona, 
University of Colorado Boulder, 
University of Oxford, University of 
Portsmouth, University of Utah, 
University of Virginia, University 
of Washington, University of 
Wisconsin, Vanderbilt University, 
and Yale University.

This work has made use of data from the European Space Agency (ESA) mission
{\it Gaia} (\url{https://www.cosmos.esa.int/gaia}), processed by the {\it Gaia}
Data Processing and Analysis Consortium (DPAC,
\url{https://www.cosmos.esa.int/web/gaia/dpac/consortium}). Funding for the DPAC
has been provided by national institutions, in particular the institutions
participating in the {\it Gaia} Multilateral Agreement.

% From the Center for Belonging and Social Change, https://cbsc.osu.edu/about-us/land-acknowledgement
We would like to acknowledge the land that The Ohio State University occupies is the ancestral and contemporary territory of the Shawnee, Potawatomi, Delaware, Miami, Peoria, Seneca, Wyandotte, Ojibwe and many other Indigenous peoples. Specifically, the university resides on land ceded in the 1795 Treaty of Greeneville and the forced removal of tribes through the Indian Removal Act of 1830. As a land grant institution, we want to honor the resiliency of these tribal nations and recognize the historical contexts that has and continues to affect the Indigenous peoples of this land.

\software{\vice \citep{johnson_impact_2020}, Astropy \citep{astropy_collaboration_astropy_2013,astropy_collaboration_astropy_2018,astropy_collaboration_astropy_2022}, scikit-learn \citep{pedregosa_scikit-learn_2011}, SciPy \citep{virtanen_scipy_2020}, Matplotlib \citep{hunter_matplotlib_2007}, NumPy \citep{harris_array_2020}, pandas \citep{reback_pandas-devpandas_2021,the_pandas_development_team_pandas-devpandas_2025}.}

\appendix

\section{Reproducibility}
\label{app:reproducibility}

The figures, tables, and models in this paper are reproducible. The git repository associated to this study is publicly available at \url{\GitHubURL}, and the release {\tt v1.0.0} allows anyone to re-build the entire manuscript. The multi-zone model outputs used to produce the figures in this work are stored at \url{https://doi.org/10.5281/zenodo.16649938}.

\bibliography{references}

\end{document}
