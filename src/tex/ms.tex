% For TeXworks
\makeatletter
\declare@file@substitution{revtex4-1.cls}{revtex4-2.cls}
\makeatother

% Define document class
\documentclass[twocolumn,twocolappendix,linenumbers]{aastex631}

% \usepackage{showyourwork}
\usepackage{amsmath}
\usepackage{amssymb}
\usepackage{graphicx}
% \usepackage{layouts}
\usepackage{xcolor}
% \usepackage{upgreek}
\let\tablenum\relax
\usepackage{siunitx}
\usepackage{xspace}

% user-defined commands
% \newcommand{\yes}{\textcolor{green}{\checkmark}\xspace}
% \newcommand{\meh}{\textcolor{black}{$\sim$}\xspace}
% \newcommand{\no}{\textcolor{red}{$\times$}\xspace}
\newcommand{\osuaffil}{Department of Astronomy, The Ohio State University, 140 W. 18th Ave, Columbus OH 43210, USA}
\newcommand{\ccappaffil}{Center for Cosmology and AstroParticle Physics, The Ohio State University, 191 W. Woodruff Ave., Columbus OH 43210, USA}
\newcommand{\aFe}{[$\alpha$/Fe]\xspace}
\newcommand{\vice}{{\tt VICE}\xspace}
% \newcommand{\hydro}{{\tt h277}\xspace}
\newcommand{\todo}[1]{{\color{red}#1}}

% Citation aliases
\defcitealias{johnson_stellar_2021}{J21}
\defcitealias{leung_variational_2023}{L23}
\defcitealias{dubay_galactic_2024}{D24}

\shorttitle{Two-Infall in VICE}
\shortauthors{Dubay et al.}
% \linespread{1.8}

\begin{document}

% Title
\title{A Two-Infall Star Formation History with Outflows and Radial Migration}

% Author list
\author[0000-0003-3781-0747]{Liam O.\ Dubay}
\affiliation{\osuaffil}
\affiliation{\ccappaffil}
\author[0000-0001-7258-1834]{Jennifer A.\ Johnson}
\affiliation{\osuaffil}
\affiliation{\ccappaffil}
\author[0000-0002-6534-8783]{James W.\ Johnson}
\affiliation{Observatories of the Carnegie Institution for Science, 813 Santa Barbara St., Pasadena CA 91101, USA}
\affiliation{\osuaffil}
\affiliation{\ccappaffil}
% \author[0000-0001-7775-7261]{David H. Weinberg}
\author{coauthors}

\correspondingauthor{Liam O.\ Dubay}
\email{dubay.11@osu.edu}

\begin{abstract}
    Abstract.
\end{abstract}

\section{Outline}
\begin{enumerate}

    \item Introduction
    \begin{enumerate}
        \item Hook, summary of alpha-bimodality
        \item Summary of two-infall literature
        \item Summary of secular evolution models which claim alpha-bimodality
        \item Goal of the paper
    \end{enumerate}
    
    \item Methods
    \begin{enumerate}
        \item Description of GCE models
        \begin{enumerate}
            \item {\it Plot:} Two-infall parameters in one-zone models. The second infall timescale $\tau_2$ is the most important parameter for fitting the low-alpha sequence and the MDF.
        \end{enumerate}
        \item Motivate outflow prescription
        \item Discuss yields
        \begin{enumerate}
            \item {\it Plot:} Yield scaling in one-zone models. No yield set is fully compatible with the observed abundance history, so we investigate both the high yield and low yield cases in our multi-zone models.
        \end{enumerate}
        \item Describe data sources
    \end{enumerate}
    
    \item Multi-zone model results
    \begin{enumerate}
        \item Stellar abundance distributions
        \begin{enumerate}
            \item {\it Plot:} Stellar [O/Fe] distributions from multi-zone models with the low and high yield sets. Using the higher yields, the global [O/Fe] distribution is triple-peaked which does not fit the data.
            \item {\it Plot:} Abundance diagram from a high-yield one-zone model which illustrates that the turn-over in [O/Fe] evolution due to the second infall epoch produces an intermediate [O/Fe] peak in the stellar distribution.
        \end{enumerate}
        \item The radial abundance gradient
        \begin{enumerate}
            \item {\it Plot:} Radial abundance gradients predicted by multi-zone models with low and high yields.
            \item A variable second infall timescale
            \item Altering the outflow prescription
            \item Radial gas flows
        \end{enumerate}
        \item Abundance histories
        \begin{enumerate}
            \item {\it Plot:} Comparison of stellar age-abundance diagrams from multi-zone models with low and high yields. The second infall epoch produces substantial dilution of the ISM metallicity, for which there is little observational evidence. Higher yields help the system recover from perturbations more rapidly and are in less disagreement with the data.
            \item {\it Plot:} Similar, but showing the mode of the stellar abundance distributions over time compared to the data. Due to the dilution from the second infall, the Solar neighborhood spends less time in equilibrium in the high-yield case, and doesn't yet reach equilibrium in the low-yield case.
            \item Can CGM pre-enrichment mitigate dilution and/or approach to equilibrium issues?
            \item Other parameters that can mitigate these issues
        \end{enumerate}
    \end{enumerate}
    
    \item Discussion
    \begin{enumerate}
        \item Other ways to fix the thin disk abundances
        \begin{enumerate}
            \item {\it Plot:} The DTD in one-zone models. A prompt DTD can reduce the size of the low-alpha loop, but it messes up the high-alpha sequence.
            \item {\it Plot:} Adjusting the thick-to-thin disk prescription in one-zone models. If the mass in the first infall is significantly more substantial, the [O/Fe] distribution better matches the data in the high yield case.
        \end{enumerate}
        \item SF hiatus not driven by infall history
        \begin{enumerate}
            \item {\it Plot:} An SFE-driven hiatus in a one-zone model can produce alpha-bimodality without altering the gas inflow prescription.
        \end{enumerate}
    \end{enumerate}
    
    \item Conclusions
    \begin{enumerate}
        \item Summary of findings
        \begin{enumerate}
            \item Attempts to fix the problem with the [O/Fe] distribution in the thin disk end up creating issues with other observables, such as the thick/thin disk surface density ratio, the SN Ia DTD, or the age-metallicity relation.
            \item The equilibrium model, if accurate, places very strict limits on the two-infall model, moreso than for other models.
            \item A short ($\sim200$ Myr) star formation hiatus early in the history of the disk is sufficient to produce a bimodal distribution of [O/Fe] without altering the gas infall prescription. This is distinct from the star formation hiatus produced by the two-infall model.
        \end{enumerate}
    \end{enumerate}
\end{enumerate}

\section{Introduction}

% To add: Carme Gallart (2019) three-infall-ish, Kobayashi metallicity limit for SNe Ia and sub-Chandra

The two-infall model has long been proposed to explain the formation of the Milky Way's thick and thin disks. The infall history of any given region of the disk is described by two successive, exponentially declining bursts, which explain the origin of the thick and thin disks respectively.

\subsection{Summary of Two-Infall Papers}

% Summary of two-infall papers
\citet{matteucci_galactic_1989} modeled the galaxy with a two-component model, a one-zone model for the halo and concentric shell zones for the disk. They used a simple exponential gas infall (so not yet a two-infall model), where the infall timescale increased proportionally with radius. They studied cases with primordial and pre-enriched gas infall, finding that the former produces a more realistic metallicity distribution in the Solar neighborhood. Their inside-out disk evolution produced good agreement with observed abundance gradients.

\citet{chiappini_chemical_1997} presented the first two-infall model of the MW disk. They adopted an inside-out prescription for the formation of the thin disk, with the timescale of the second infall increasing linearly with radius. They assume the formation timescale of the thick disk is 1 Gyr, and the time of the second infall is 2 Gyr. They also assume a lower threshold on the gas density required for star formation, which produces a star formation hiatus at the end of the thick disk phase. They also noted the ``loop'' that appears in the \aFe--[Fe/H] relation during the thin disk phase. They mostly compare to Solar neighborhood data.

\citet{chiappini_abundance_2001} updated their model to fit the observed abundance, stellar, and gas gradients in the Milky Way disk. They find that a density threshold for star formation, making a gap in star formation between the halo and disk phases, is necessary to reproduce observational data. They argue that the details of halo formation also have a strong impact on the outer disk. No radial flows or outflows. I'm confused about their distinction between halo and thick disk, if any.

\citet{chiappini_oxygen_2003} compare model predictions of C, N, and O to Milky Way and other spiral galaxy data. They predict a very slow enrichment between the birth of the Sun and present day, which they claim is due to their adopted gas density threshold for star formation (the infall rate is low enough that the SFR oscillates from timestep to timestep).

\citet{matteucci_effect_2009} investigated different SN Ia delay time distributions using the \citep{chiappini_chemical_1997,chiappini_abundance_2001} model. They use a model for the full disk but compare only to Solar neighborhood data. They concluded that some prompt SN Ia enrichment is necessary to match observed abundance patterns.

\citet{spitoni_effects_2009} relaxed the instantaneous recycling approximation in two different ways: galactic fountains (i.e., SN ejecta are launched out of the disk and take some time to return) and metal cooling (i.e., newly-enriched hot gas takes time to cool before it can form stars). They found that fountains do not strongly affect the evolution of the $\alpha$-elements for short (few hundred Myr) timescales. The effect of a metal cooling timescale is similarly negligible if the yields are not dependent on metallicity.

\citet{spitoni_effects_2011} find that a two-infall model of the disk without gas exchange produces a radial metallicity gradient which is too shallow. They implement an inward radial gas flow on the order of $\sim0-4$ km s$^{-1}$ which varies with radius, and find that it improves agreement with the observed gradient. However, they found that a variable star formation efficiency with radius in combination with a gas density threshold for star formation could also reproduce the observed gradient without radial flows.

\citet{spitoni_effect_2015} add a simple radial migration scheme to the model of \citet{spitoni_effects_2011}. They adopt stellar velocities of $\sim 1$ km s$^{-1}$ and assume that some fixed fraction of stars born at a given radius will end up in the Solar vicinity (10\% of those born at 4 kpc and 20\% at 6 kpc). This mimics the results of previous dynamical models such as \citet{minchev_chemodynamical_2013}. They find that by including stellar migration, they are better able to reproduce the high-metallicity tail of the local metallicity distribution, and they also argue that they constrain the migration speed to $0.5 < v < 2$ km s$^{-1}$ based on this tail. Their only point of comparison with data is the local G-dwarf metallicity distribution.

\citet{grisoni_ambre_2017} argue for a ``parallel'' scenario, where the thick and thin disks form simultaneously but with very different infall timescales (0.1 and 7 Gyr, respectively). They compare their models with stellar abundances from the AMBRE sample of solar-neighborhood dwarfs. They argue that the parallel scenario can produce metal-rich high-alpha stars in situ, whereas in the two-infall scenario these stars would have to migrate from the inner disk.

\citet{palla_chemical_2020} explored radial gas flows, variable star formation efficiency of the thin disk, and the onset time of the second infall (but not the infall timescales). They find that inside-out formation (i.e., longer infall timescales in the outer disk) is not sufficient to reproduce observed abundance gradients, and that radial gas flows, possibly in combination with a radially-varying star formation law, are key. Pre-enriched infall in the inner disk ($R<8$ kpc) at the level of -0.5 dex also improves agreement with data.

% Radial gas flows
Some two-infall studies \citep[e.g.,][]{spitoni_effects_2011,palla_chemical_2020,palla_mapping_2024} implement inward radial gas flows with velocity $\sim1$ km s$^{-1}$ in order to reproduce the radial abundance gradient without Galactic outflows.

% Three-infall
Motivated by constraints on the SFH by {\it Gaia}, \citet{spitoni_beyond_2023} proposed the three-infall model, an extension with a third phase of exponential gas infall in the last few Gyr.

\citet{palla_mapping_2024} compare model predictions to abundances and ages of open clusters from the {\it Gaia}-ESO survey. They find that the two-infall model over-predicts the metallicity of the youngest clusters, a problem which is fixed by the three-infall model. A version of the three-infall model with milder dilution (mild pre-enrichment of the gas with a smaller amount of infall) compared to \citet{spitoni_beyond_2023} can simultaneously match the metalliticy of old, intermediate, and young clusters across the disk.

\citet{spitoni_remind_2024}

\section{Observational Sample}
\label{sec:observational-sample}

\begin{table*}
    \centering
    \caption{Sample selection parameters and median uncertainties from APOGEE DR17 (see Section \ref{sec:observational-sample}).}
    \label{tab:sample}
    \begin{tabular}{lll}
        \hline\hline
        \multicolumn{1}{c}{Parameter} & \multicolumn{1}{c}{Range or Value} & \multicolumn{1}{c}{Notes} \\
        \hline
        $\log g$            & $1.0 < \log g < 3.8$          & Select giants only \\
        $T_{\rm eff}$       & $3500 < T_{\rm eff} < 5500$ K & Reliable temperature range \\
        $S/N$               & $S/N > 80$                    & Required for accurate stellar parameters \\
        ASPCAPFLAG Bits     & $\notin$ 23                   & Remove stars flagged as bad \\
        EXTRATARG Bits      & $\notin$ 0, 1, 2, 3, or 4     & Select main red star sample only \\
        Age                 & $\sigma_{\rm Age} < 40\%$     & Age uncertainty from \citetalias{leung_variational_2023} \\
        $R_{\rm gal}$     & $3 < R_{\rm gal} < 15$ kpc    & Eliminate bulge \& extreme outer-disk stars \\
        $|z|$               & $|z| < 2$ kpc                 & Eliminate halo stars \\
        \hline
    \end{tabular}
\end{table*}

\begin{table*}
\centering
\caption{Number of APOGEE stars in each Galactic region.}
\label{tab:apogee-regions}
\begin{tabular}{r|cccccc}
\hline\hline
$R_{\rm gal}\in$ & $(3, 5]$ kpc & $(5, 7]$ kpc & $(7, 9]$ kpc & $(9, 11]$ kpc & $(11, 13]$ kpc & $(13, 15]$ kpc \\
$|z|\in$ &  &  &  &  &  &  \\
\hline
$(1.0, 2.0]$ kpc & 2013 & 2100 & 8734 & 3663 & 1324 & 363 \\
$(0.5, 1.0]$ kpc & 2487 & 3490 & 13811 & 9069 & 3289 & 460 \\
$(0.0, 0.5]$ kpc & 3296 & 7029 & 17319 & 16276 & 6336 & 812 \\
\hline
\end{tabular}

\end{table*}

\begin{table*}
    \centering
    \caption{Median and dispersion in APOGEE parameter uncertainties.}
    \label{tab:uncertainties}
    \begin{tabular}{lcc}
        \hline\hline
        \multicolumn{1}{c}{Parameter} & \multicolumn{1}{c}{Median Uncertainty} & \multicolumn{1}{c}{Uncertainty Dispersion ($95\%-5\%$)} \\
        \hline
        [Fe/H]          & $0.0089$   & $0.0060$ \\
        ${\rm [O/Fe]}$  & $0.019$    & $0.031$ \\
        log(Age/Gyr)    & $0.10$     & $0.16$ \\
        \hline
    \end{tabular}
\end{table*}

We compare our models against stellar abundances from the Apache Point Observatory Galactic Evolution Experiment \citep[APOGEE;][]{majewski_apache_2017} data release 17 \citep[DR17;][]{abdurrouf_seventeenth_2022}. APOGEE data were obtained from infrared spectrographs \citep{wilson_apache_2019} mounted on the 2.5-meter Sloan Foundation Telescope \citep{gunn_25_2006} at Apache Point Observatory and the Ir{\'e}n{\'e}e DuPont Telescope \citep{bowen_optical_1973} at Las Campanas Observatory. The data reduction pipeline is described by \citet{nidever_data_2015}, and APOGEE Stellar Parameter and Chemical Abundance Pipeline (ASPCAP) is detailed by \citet{holtzman_abundances_2015}, \citet{garcia_perez_aspcap_2016}, and \citet{jonsson_apogee_2020}.

We obtain a sample of \input{output/sample_size.txt} red giant branch and red clump stars with high-quality spectra using the selection criteria listed in Table \ref{tab:sample}, which are adapted from \citet{hayden_chemical_2015}. We use the \citet{bailer-jones_estimating_2021} photo-geometric distance estimates from {\it Gaia} Early Data Release 3 \citep{gaia_collaboration_gaia_2016,gaia_collaboration_gaia_2021} included in the APOGEE DR17 catalog. When calculating the galactocentric radius $R_{\rm gal}$ and midplane distance $z$ of each star, we assume the distance between the Sun and the Galactic center is $R_\odot=8.122$ kpc \citep{gravity_collaboration_detection_2018} and the Solar midplane distance is $z_\odot=20.8$ pc \citep{bennett_vertical_2019}. Table \ref{tab:apogee-regions} presents the number of APOGEE stars in bins of $R_{\rm gal}$ and $|z|$; typical distance uncertainties are much smaller than the bin width. \todo{Mention calibrated abundances.}

\todo{Taken from last paper.} We use estimated ages from \citet[][hereafter \citetalias{leung_variational_2023}]{leung_variational_2023}, who use a variational encoder-decoder network which is trained on asteroseismic data to retrieve age estimates for APOGEE giants without contamination from age-abundance correlations. Importantly, the \citetalias{leung_variational_2023} ages do not plateau beyond $\sim10$ Gyr as they do in astroNN \citep{mackereth_dynamical_2019}. We use an age uncertainty cut of 40\% per the recommendations of \citetalias{leung_variational_2023}, which produces a total sample of \input{output/age_sample_size.txt} APOGEE stars with age estimates. We note that we use the full sample of \input{output/sample_size.txt} APOGEE stars unless we explicitly compare to age estimates. Table \ref{tab:uncertainties} presents the median and dispersion ($95^{\rm th} - 5^{\rm th}$ percentile difference) of the uncertainty in [Fe/H], [O/Fe], and log(age).

\section{Chemical Evolution Models \& Parameter Selection}
\label{sec:methods}

\todo{Copied from last paper.} We use the multi-zone GCE model tools in \vice developed by \citet{johnson_stellar_2021}. The basic setup of our models follows theirs. The disk is divided into concentric rings of width $\delta R_{\rm gal}=100$ pc. Stellar populations migrate radially under the prescription we describe in Appendix C, but each ring is otherwise described by a conventional one-zone GCE model with instantaneous mixing. Following \citet{johnson_stellar_2021}, we do not implement radial gas flows \citep[e.g.,][]{lacey_chemical_1985,bilitewski_radial_2012}. Stellar populations are also assigned a distance from the midplane according to their age and final radius as described in Appendix C.

\todo{Copied from last paper.} We run our models with a time-step size of $\Delta t=10$ Myr up to a maximum time of $t_{\rm final}=13.2$ Gyr. Following \citet{johnson_stellar_2021}, we set \vice to form $n=8$ stellar populations per ring per time-step, and we set a maximum star-formation radius of $R_{\rm SF} = 15.5$ kpc, such that $\dot\Sigma_\star=0$ for $R_{\rm gal}>R_{\rm SF}$. The model has a full radial extent of 20 kpc, allowing a purely migrated population to arise in the outer 4.5 kpc. We adopt continuous recycling, which accounts for the time-dependent return of mass from all previous generations of stars \citep[see Equation 2 from][]{johnson_impact_2020}. We summarize these parameters in Table \ref{tab:multizone-parameters}.

\begin{deluxetable*}{Cccl}
    \tablecaption{A summary of variables and their fiducial values for our chemical evolution models (see discussion in Section \ref{sec:methods}).\label{tab:multizone-parameters}}
    \tablehead{
        \colhead{Quantity} & \colhead{Fiducial Value(s)} & \colhead{Section} & \colhead{Description}
    }
    \startdata
        % Yields
        y/Z_\odot       & 1         & \ref{sec:yields}  & Nucleosynthetic yield scaling (see Table \ref{tab:yields}) \\
        N_{\rm Ia}/M_\star  & $1.55\times10^{-3}$       & \ref{sec:yields}  & Integrated Type Ia supernova rate (see Table \ref{tab:yields}) \\
        f_{\rm Ia}(t)   & Plateau   & \ref{sec:yields}  & Delay-time distribution of Type Ia supernovae \\
        % t_D             & 40 Myr    & \ref{sec:yields}  & Minimum SN Ia delay time \\
        % Star formation histories
        % Outflows
        \eta_\odot      & 0.2       & \ref{sec:outflows}    & Outflow mass-loading factor in the Solar annulus ($\eta\equiv \dot\Sigma_{\rm out}/\dot\Sigma_\star$) \\
        R_\eta          & 4.2 kpc   & \ref{sec:outflows}    & Exponential outflow scale radius \\
        % f_{\rm in}(t|R_{\rm gal})   & Equation \ref{eq:infall-rate}   & \ref{sec:sfh} & Time-dependence of the gas infall rate \\
        \tau_1          & 1 Gyr     & \ref{sec:sfh}     & Timescale of the first infall epoch \\
        \tau_2          & 15 Gyr    & \ref{sec:sfh}     & Timescale of the second infall epoch \\
        t_{\rm max}      & 4.2      & \ref{sec:sfh}     & Time of maximum gas infall (onset of second infall) \\
        f_{\Sigma,0}    & 0.27      & \ref{sec:sfh}     & Central thick/thin disk surface density ratio \\
        {\rm [X/H]}_{\rm CGM}   & Pristine  & \ref{sec:sfh}     & Metallicity of infalling gas \\
        \sigma_{\rm RM8}    & 2.68 kpc  & \ref{sec:migration}   & Radial migration strength
    \enddata
\end{deluxetable*}
\vspace{-24pt}

\subsection{Nucleosynthetic Yields}
\label{sec:yields}

The population-averaged nucleosynthetic yields of CCSNe are still fairly uncertain. This problem is exacerbated by the complexity of the CCSN explosion landscape \citep{sukhbold_core-collapse_2016}. Recently, \citet{weinberg_scale_2024} used a measurement of the mean Fe yield of CC SNe by \citet{rodriguez_iron_2023} and the observed \aFe plateau at low metallicity to infer population-averaged yields of $y/Z_\odot\approx1$ --- in other words, the scale of SN yields relative to the Solar abundance is of order unity.

\begin{table}
    \centering
    \caption{Nucleosynthetic yield sets.}
    \begin{tabular}{c|cc}
\hline\hline
Element & $y_{\rm X}^{\rm CC}$ & $y_{\rm X}^{\rm Ia}$ \\
\hline
O & \num{5.91e-03} & \num{0} \\
Mg & \num{7.29e-04} & \num{0} \\
Si & \num{5.58e-04} & \num{1.44e-04} \\
Fe & \num{4.73e-04} & \num{7.74e-04} \\
\hline
\end{tabular}

    \label{tab:yields}
\end{table}

We increased $y_{\rm Fe}^{\rm Ia}$ by a factor of 30\% and 10\% for the {\tt yZ1} and {\tt yZ2} yield sets, respectively, to ensure our models could reach ${\rm [O/Fe]}\approx0.0$ by the end time. The final row of Table \ref{tab:yields} reports the integrated SN Ia rate $N_{\rm Ia}/M_\star$ from each yield set, assuming a mean Fe yield per SN Ia of $\overline m_{\rm Fe}^{\rm Ia}=0.7$ M$_\odot$ \citep{mazzali_common_2007,howell_effect_2009} \todo{(are there more recent citations?)}. The rate for the {\tt yZ1} yield set is slightly higher than the volumetric rate of $N_{\rm Ia}/M_\star=(1.3\pm0.1)\times10^{-3}\,{\rm M}_\odot^{-1}$ reported by \citet{maoz_star_2017}, but is consistent with their measurement of $N_{\rm Ia}/M_\star=(1.6\pm0.3)\times10^{-3}\,{\rm M}_\odot^{-1}$ for field galaxies. The rate for the {\tt yZ2} yield set is consistent with the measurement of $N_{\rm Ia}/M_\star=(2.2\pm1.0)\times10^{-3}\,{\rm M}_\odot^{-1}$ by \citet{maoz_type-ia_2012}.

The SN Ia rate in units of ${\rm M}_\odot^{-1}\,{\rm yr}^{-1}$ is defined as
\begin{equation}
    R_{\rm Ia}(t) = 
    \begin{cases}
        \frac{N_{\rm Ia}}{M_\star}
        \frac{f_{\rm Ia}(t)}{\int_{t_D}^{t_{\rm max}} f_{\rm Ia}(t') dt'}, & t \ge t_D \\
        0 & t < t_D,
    \end{cases}
    \label{eq:dtd-function}
\end{equation}
where $t_D=40$ Myr is the minimum SN Ia delay time, $t_{\rm max}=13.2$ Gyr is the lifetime of the disk, $N_{\rm Ia}/M_\star$ is the total number of SNe Ia per ${\rm M}_\odot$ of stars formed, and $f_{\rm Ia}(t)$ is the un-normalized form of the DTD. Motivated by the results of \citet{dubay_galactic_2024}, we adopt a wide plateau DTD of the form
\begin{equation}
    \label{eq:plateau-dtd}
    f_{\rm Ia}(t) =
    \begin{cases}
        1, & t < 1\,{\rm Gyr} \\
        (t/1\,{\rm Gyr})^{-1.1}, & t \ge 1\,{\rm Gyr}.
    \end{cases}
\end{equation}
We discuss the implications of using a different DTD in Section \ref{sec:abundance-distributions}.

\todo{Compare these yields to previous studies.}

Figure \ref{fig:yield-outflow} illustrates the effect of the yield scaling on the abundance evolution in one-zone models. All models feature a rapid dilution of the ISM metallicity by $\sim0.5-0.8$ dex, visible in the top two panels, brought on by the infall of pristine gas at $t_{\rm max}$. For the model with $y/Z_\odot=1$, this dilution persists for some time and the metallicity does not return to Solar until the present day. The models with higher yields and outflows recover from this dilution more quickly, returning to Solar metallicity by $\sim5$ Gyr ago. However, the high-yield models experience a decline in [O/Fe] of $\sim0.2$ dex between the second infall and the present day, contrasted with the smaller decline of $\sim0.1$ dex in the low-yield model.

Figure \ref{fig:yield-outflow} also indicates the mode of the APOGEE abundance distributions in 1 Gyr-wide age bins. As explained by \citet{johnson_milky_2024}, the mode is expected to be less sensitive to the effects of radial migration than other statistical measures. The data show that the evolution in [O/H] is close to flat over the past 5 Gyr. The behavior of the $y/Z_\odot=2$ and $y/Z_\odot=3$ models closely matches this trend in the data, whereas the $y/Z_\odot=1$ model shows significant evolution of $\sim0.2$ dex during the same time period. The [Fe/H] abundance in the data does increase slightly at late times, likely due to the delayed contribution of Fe from SNe Ia. Between lookback times of $\sim5-9$ Gyr, the modes of [O/H] and [Fe/H] are actually higher than the present-day, likely due to a larger population of migrated stars relative to stars born in-situ at those times. The [O/Fe] evolution of all three models is nearly identical over the past 5 Gyr, and shows a similar evolution to the data apart from a $\sim0.05$ dex offset. \todo{Is this offset a problem? I could raise the SN Ia yields, but the global distributions would not match quite as well.}

\begin{figure}
    \centering
    \includegraphics{figures/yield_outflow.pdf}
    \caption{The abundance evolution of three one-zone models with different yield and outflow settings. Table \ref{tab:yields} presents the population-averaged yields for each model. The gray points with error bars indicate the mode of the APOGEE abundance distributions from the Solar neighborhood ($7\leq R_{\rm gal}\leq 9$ kpc, $0\leq|z|\leq0.5$ kpc) in 1 Gyr-wide bins of the \citet{leung_variational_2023} age estimates.}
    \label{fig:yield-outflow}
\end{figure}

\subsection{Outflows}
\label{sec:outflows}

Most of the previous two-infall studies assume no mass-loaded outflows. Even in studies such as \citet{palicio_analytic_2023} which do incorporate Galactic winds, they are relatively weak ($\eta\approx0.2$). 

\todo{Motivate our use of outflows. Connect to James' equilibrium paper.}

\citet{weinberg_equilibrium_2017} showed that in the limit of continuous star formation, the ISM abundance approaches an equilibrium which is determined by the balance of yields, outflows, and star formation. We calculate the mass-loading factor required to reach equilibrium at the Solar abundance as
\begin{equation}
    \eta_\odot = y_{\rm O}^{\rm CC} / Z_{\rm O,\odot} - 1 + r + \tau_\star / \tau_{\rm SFH},
\end{equation}
where $r=0.4$ is the instantaneous recycling parameter, $\tau_\star$ is the star formation efficiency timescale, and $\tau_{\rm SFH}$ is the star formation timescale.

We adopt a fiducial prescription for the outflow mass-loading factor which increases exponentially with radius:
\begin{equation}
    \eta(R_{\rm gal}) = \eta_\odot \exp\Big(\frac{R_{\rm gal}-R_\odot}{R_\eta}\Big)
\end{equation}
where $R_\odot=8$ kpc. As discussed by \citet{johnson_milky_2024}, the exponential outflow scale radius $R_\eta$ sets the equilibrium abundance gradient $\nabla{\rm [O/H]}_{\rm eq}$:
\begin{equation}
    R_\eta = -\frac{1}{\nabla{\rm [O/H]}_{\rm eq} \times\ln(10)}.
\end{equation}
We adopt $R_\eta=7.2$ kpc as our fiducial value, which corresponds to $\nabla{\rm [O/H]}_{\rm eq}=-0.06$ dex kpc$^{-1}$.
\todo{Motivate exponential scale radius.}

\subsection{The Gas Supply}
\label{sec:sfh}

We run {\tt VICE} in ``infall mode,'' where we specify the gas infall density $\dot\Sigma_{\rm in}$ and the star formation effieciency (SFE) timescale $\tau_\star\equiv \Sigma_g / \dot\Sigma_\star$ as functions of time. The gas surface density $\Sigma_g$ and star formation rate $\dot\Sigma_\star$ are calculated from the two specified quantities according to our star formation law, which is described in Section \ref{sec:sf-law}, assuming zero initial gas mass in all zones.

The infall rate as a function of time and galactocentric radius can generically be described by
\begin{equation}
    \label{eq:infall-rate}
    \dot\Sigma_{\rm in}(t,R_{\rm gal}) = A f_{\rm in}(t|R_{\rm gal}) g(R_{\rm gal}),
\end{equation}
where $g(R_{\rm gal})=\Sigma_\star(R_{\rm gal}) / \Sigma_\star(R_{\rm gal}=0)$ is the stellar density gradient, $f_{\rm in}$ is the infall rate over time, and $A$ is the normalization. Because we incorporate mass-loaded outflows, $A$ is not analytically solvable, so first we numerically integrate the star formation rate $\dot\Sigma_\star(t,R_{\rm gal})$ and then follow the procedure outlined in Appendix B of \citet{johnson_stellar_2021} to calculate $A$. The infall rate is normalized to produce a total disk stellar mass of $(5.17\pm1.11)\times 10^{10}\,{\rm M}_\odot$ \citep{licquia_improved_2015} and to match the stellar surface density gradient of \citet{bland-hawthorn_galaxy_2016}.

The infall rate is described by two successive, exponentially declining bursts in time. The first infall component induces the formation of the thick disk, and the second component produces the thin disk. At a given galactocentric radius $R_{\rm gal}$, the un-normalized form of the infall rate is
\begin{equation}
    \label{eq:twoinfall-ifr}
    f_{\rm in}(t|R_{\rm gal}) = e^{-t/\tau_1} + f_{2/1}(R_{\rm gal}) e^{-(t-t_{\rm max})/\tau_2},
\end{equation}
where $\tau_1$ and $\tau_2$ are the first and second infall timescales, respectively, $t_{\rm max}$ is the onset of the second infall and thus the time of maximum gas infall, and $f_{2/1}$ is the ratio of the second infall amplitude to the first. We calculate $f_{2/1}$ for each zone such that the resulting stellar density profile follows a two-component disk, with the surface density ratio of the thick and thin disks given by
\begin{equation}
    f_\Sigma(R) \equiv \frac{\Sigma_1(R)}{\Sigma_2(R)} = f_{\Sigma,0} e^{R(1/R_2 - 1/R_1)}.
\end{equation}
We adopt a thick disk scale radius of $R_1=2.0$ kpc, a thin disk scale radius of $R_2=2.5$ kpc, and a central surface density ratio of $f_{\Sigma,0}=0.27$ \citep{bland-hawthorn_galaxy_2016}.

\begin{figure*}
    \centering
    \includegraphics{figures/onezone_params.pdf}
    \caption{Gas abundance tracks in the [O/Fe]--[Fe/H] plane for one-zone chemical evolution models which assume different values for the infall history parameters. In each panel, one parameter is varied according to the legend while the other two are held fixed. The open symbols along each curve mark logarithmic steps in time, as denoted in panel (b). The marginal panels show the corresponding stellar abundance distributions, which are convolved with a Gaussian kernel with a width of 0.02 dex for visual clarity. All models use the {\tt yZ1} yield set and assume $\eta=0.4$.}
    \label{fig:twoinfall-parameters}
\end{figure*}

% The parameters of Equation \ref{eq:infall-rate} have important effects on the chemical evolution of the model. 
% Previous studies of the two-infall model have adopted values for the infall parameters of $\tau_1\sim0.1-1$ Gyr, $\tau_2\sim3-10$ Gyr, and $t_{\rm max}\sim3-5$ Gyr. 
Figure \ref{fig:twoinfall-parameters} illustrates the effect of varying the infall parameters on gas abundance tracks and stellar abundance distributions in a one-zone model. The first infall timescale $\tau_1$, shown in panel (a), primarily affects the stellar distribution along the high-$\alpha$ sequence. Though $\tau_1$ has an apparently large effect on the size of the low-$\alpha$ loop, the effect on the stellar abundance distribution of the low-$\alpha$ sequence is quite small due to the low number of stars formed between $t\sim3-6$ Gyr. We adopt $\tau_1=1$ Gyr for our fiducial value, in line with \citet{spitoni_galactic_2020} but longer than, e.g., \citet{nissen_high-precision_2020} or \citet{spitoni_apogee_2021}, in order to set the peak of the high-$\alpha$ sequence at ${\rm [O/Fe]}\approx+0.3$. 

Panel (b) of Figure \ref{fig:twoinfall-parameters} shows that 
% the second infall timescale $\tau_2$ is the most important parameter to tune in order to reproduce the observed stellar abundance distributions. 
the second infall timescale $\tau_2$ controls the width of the MDF and the low-$\alpha$ [O/Fe] distribution. A shorter $\tau_2$ produces a broader [O/Fe] distribution which is skewed to higher [O/Fe], while a longer $\tau_2$ produces both a narrower low-$\alpha$ sequence and a narrower MDF. We note that our maximum value of $\tau_2=30$ Gyr is very close to a constant infall rate, so a further increase in $\tau_2$ has diminishing returns. Between $\tau_2=3-30$ Gyr, the endpoint of the abundance tracks shifts by $\sim0.2$ dex in [Fe/H] and $\sim0.1$ dex in [O/Fe], which could affect the model's ability to reproduce the present-day abundance of the Solar neighborhood. We adopt a fiducial value of $\tau_2=15$ Gyr for the Solar neighborhood in order to minimize the width of the low-$\alpha$ [O/Fe] distribution while still approaching Solar [Fe/H] at late times. This value is in line with the infall timescale recovered by \citet{spitoni_galactic_2020}, and similar to the local star formation timescale adopted by \citet{johnson_stellar_2021}, but significantly longer than the timescales found by \citet{nissen_high-precision_2020} and \citet{spitoni_apogee_2021}. In \todo{Section X}, we explore the effect of varying $\tau_2$ with radius in multi-zone models.

Finally, panel (c) of Figure \ref{fig:twoinfall-parameters} shows that the time of maximum infall $t_{\rm max}$ (c) strongly affects the overall stellar abundance distribution for values $t_{\rm max}\leq2$ Gyr, but in this case the gas tracks do not produce the characteristic abundance loop. For $t_{\rm max}>2$ Gyr, varying $t_{\rm max}$ results in a minor shift to the mean of the MDF and little change to the [O/Fe] distributions, even though the abundance tracks in [O/Fe]--[Fe/H] space appear very different. The value of $t_{\rm max}$ also slightly adjusts the ISM abundance endpoint, as a longer $t_{\rm max}$ means the chemical evolution ``reset'' from the second infall occurs closer to the present day (see discussion in Section \ref{sec:age-abundance}. We adopt a fiducial value of $t_{\rm max}=4.2$ Gyr so that the second infall occurs at a lookback time of 9 Gyr, younger than the median age of the thick disk in the APOKASC-3 catalog of $9.14\pm0.05$ Gyr \citep{pinsonneault_apokasc-3_2024}. This value is generally in line with previous two-infall studies \citep[e.g.,][]{nissen_high-precision_2020,spitoni_galactic_2020,spitoni_apogee_2021}.

We note that all our models are normalized to produce the same thick-to-thin-disk mass ratio regardless of the infall parameters. The high-$\alpha$ sequence appears much less prominent in our [O/Fe] distributions in Figure \ref{fig:twoinfall-parameters} than in the data because the model outputs include only stars which were formed in-situ at the Solar annulus. In our multi-zone models, most of the high-$\alpha$ stars present in the Solar neighborhood have migrated from the inner Galaxy.

In most of our models, we assume the infalling gas is pristine (i.e., $Z_{\rm in}=0$). However, the circumgalactic medium (CGM) from which the infalling gas is drawn could be previously enriched, either from gas stripped from dwarf galaxies or from SNe in the halo \todo{(citations; any measurements of CGM metallicity?)}. For this reason, we also test cases where the infalling gas is pre-enriched and its metallicity is described by
\begin{equation}
    \label{eq:pre-enrichment}
    Z_{\rm in}(t) = (1 - \exp(-t/\tau_{\rm rise})) Z_\odot 10^{{\rm [X/H]}_{\rm CGM}}.
\end{equation}
In this case, the metallicity rises from 0 with a timescale $\tau_{\rm rise}=2$ Gyr and plateaus at ${\rm [X/H]}_{\rm CGM}={\rm [O/H]}_{\rm CGM}={\rm [Fe/H]}_{\rm CGM}$. We also investigate cases where the CGM is $\alpha$-enhanced, i.e., ${\rm [O/Fe]}_{\rm CGM}>0$.

\begin{figure}
    \centering
    \includegraphics{figures/onezone_sfr.pdf}
    \caption{Key takeaway: a fundamental attribute of any two-infall model is a bump in the number of stars on the low-alpha sequence where the abundance track turns over. This is because the star formation rate is near its peak while the rate of chemical evolution has slowed considerably.}
    \label{fig:onezone-sfr}
\end{figure}

\subsection{The Star Formation Law}
\label{sec:sf-law}

The star formation law follows a single power-law prescription: $\dot\Sigma_\star\propto\Sigma_g^N$, with $N=1.5$ following \citet{kennicutt_global_1998}. Previous work with this GCE model \citep[e.g.,][]{johnson_stellar_2021,dubay_galactic_2024} assumed a three-component power-law, but we adopt a single power-law prescription in this work to allow for a more direct comparison with previous two-infall studies \citep[e.g.,][]{spitoni_remind_2024}. In detail, we calculate the star formation efficiency (SFE) timescale $\tau_\star\equiv\Sigma_g/\dot\Sigma_\star$ according to the following:
\begin{equation}
    \label{eq:sf-law}
    \tau_\star = 
    \begin{cases}
        \varepsilon(t) \tau_{\rm mol}(t),   & \Sigma_g \ge \Sigma_{g,0} \\
        \varepsilon(t) \tau_{\rm mol}(t) \Big(\frac{\Sigma_g}{\Sigma_{g,0}}\Big)^{-1/2}, & \Sigma_g < \Sigma_{g,0}
    \end{cases}
\end{equation}
where $\Sigma_{g,0} = 10^8\,{\rm M}_\odot\,{\rm kpc}^{-2}$ and $\tau_{\rm mol}(t)=\tau_{\rm mol,0}(t/t_0)^\gamma$ with $\gamma=1/2$ and $\tau_{\rm mol,0}=2$ Gyr. Previous two-infall studies \citep[e.g.,][]{nissen_high-precision_2020} have adopted a higher SFE during the first infall epoch than during the second, which we emulate through the pre-factor $\varepsilon$:
\begin{equation}
    \label{eq:sfe-prefactor}
    \varepsilon(t) = 
    \begin{cases}
        0.5, & t < t_{\rm max} \\
        1.0, & t \ge t_{\rm max}.
    \end{cases}
\end{equation}
A lower value of $\varepsilon(t<t_{\rm max})$ leads to more efficient star formation during the first infall epoch. Figure \ref{fig:sfe-prefactor} illustrates that this pre-factor largely affects the metallicity of the high-$\alpha$ sequence, with a smaller $\varepsilon$ producing faster enrichment during the first infall and stronger dilution after $t_{\rm max}$. The pre-factor has virtually no effect on the overall [O/Fe] distribution because the model is normalized to produce the same thick-to-thin-disk mass ratio regardless of the details of the star formation law. We adopt $\varepsilon(t<t_{\rm max})=0.5$ for consistency with the two-infall literature.

\begin{figure}
    \centering
    \includegraphics{src/tex/figures/sfe_prefactor.pdf}
    \caption{Effect of the SFE timescale pre-factor $\varepsilon$ on abundance tracks and distributions in a one-zone model (see Section \ref{sec:sf-law}). All models are normalized to produce roughly the same ratio of thick to thin disk stars regardless of the value of $\varepsilon$ during the first infall epoch.}
    \label{fig:sfe-prefactor}
\end{figure}

\subsection{Stellar Migration}
\label{sec:migration}

The distance a stellar population born at $R_{\rm form}$ migrates over its age $\tau$ is drawn from a Gaussian centered at 0 with standard deviation
\begin{equation}
    \sigma_{\rm RM} = \sigma_{\rm RM8} \Big(\frac{\tau}{8\,{\rm Gyr}}\Big)^{0.33} \Big(\frac{R_{\rm form}}{8\,{\rm kpc}}\Big)^{0.61},
    \label{eq:radial-migration}
\end{equation}
where we adopt $\sigma_{\rm RM8}=2.68$ kpc as the fiducial value for the strength of radial migration. This is smaller than the value of $\sigma_{\rm RM8}=3.6$ kpc found by \citet{frankel_measuring_2018}, but in \todo{Section X} we explore the effect of a stronger migration prescription.

All stellar populations are born at the Galactic midplane and are assigned a final midplane distance $z$ drawn from the distribution
\begin{equation}
    p(z|\tau,R_{\rm final}) = \frac{1}{4 h_z} {\rm sech}^2\Big(\frac{z}{2 h_z}\Big),
    \label{eq:sech-pdf}
\end{equation}
where $R_{\rm final}$ is the final Galactocentric radius of the stellar population. The width of the distribution $h_z$ is given by
\begin{equation}
    h_z(\tau,R_{\rm final}) = \Big(\frac{0.24\,{\rm kpc}}{e^2}\Big) \exp\Big(\frac{\tau}{7\,{\rm Gyr}} + \frac{R_{\rm final}}{6\,{\rm kpc}}\Big).
    \label{eq:scale-height}
\end{equation}
We note that the final midplane distance is assigned at the end of the model run and therefore does not affect the chemical evolution.

The parameters of Equations \ref{eq:radial-migration} and \ref{eq:scale-height} were chosen to fit the stellar migration patterns in the {\tt h277} hydrodynamical simulation \citep{christensen_implementing_2012}. A more complete discussion of the migration scheme and its consequences can be found in Appendix C of \citet{dubay_galactic_2024}.

\section{Multi-Zone Model Results}
\label{sec:multizone-results}

\subsection{Present-Day Abundances}
\label{sec:abundance-distributions}

\begin{figure*}
    \centering
    \includegraphics{src/tex/figures/ofe_df_comparison.pdf}
    \caption{Stellar [O/Fe] distributions produced by multi-zone models (a--d) and as observed by APOGEE (e). Stars are binned by Galactocentric radius, represented by the color scale, and absolute midplane distance $|z|$, represented by the different rows. Each distribution is normalized so that the area under the curve is 1, and the vertical scale is consistent across each row. The median APOGEE abundance uncertainties were forward-modeled onto the model outputs. For visual clarity, each distribution is smoothed with a box-car of width 0.05 dex. Model (a) assumes the fiducial parameters and the {\tt yZ1} yield set, whereas models (b--d) assume the {\tt yZ2} yield set. Model (c) also assumes a power-law SN Ia DTD, while model (d) assumes the infalling gas has metallicity ${\rm [O/H]}_{\rm CGM}={\rm [Fe/H]}_{\rm CGM}=-0.7$. The distributions from APOGEE DR17 are plotted in column (e) for reference.}
    \label{fig:ofe-df}
\end{figure*}

As discussed in Section \ref{sec:sfh}, the two-infall model generically predicts {\it three} peaks in the [O/Fe] distribution which correspond to the high-$\alpha$ sequence, the abundance ``turn-over'' after the second infall, and finally the late-time low-$\alpha$ sequence. Figure \ref{fig:ofe-df} compares [O/Fe] distributions across the Galactic disk produced by four different multi-zone models with the distributions observed in APOGEE DR17. For the fiducial model (a) with $y/Z_\odot=1$, the latter two peaks are close together and the overall effect is similar to the observed distribution. With a higher yield set, however, there is a $\sim0.2$ dex separation between the low- and intermediate-$\alpha$ peaks, because the CCSN element production after the second infall is more efficient. As a result, model (b) predicts a high density of stars in the APOGEE ``trough.''

In columns (c) and (d) of Figure \ref{fig:ofe-df}, we attempt to mitigate the issue of the intermediate-$\alpha$ peak for the $y/Z_\odot=2$ yield set. Model (c) switches out our fiducial DTD with a power-law DTD, which decreases the timescale for Fe production (see Figure \ref{fig:dtd}). However, as shown by \citet{dubay_galactic_2024}, models with a power-law DTD struggle to reproduce the high-$\alpha$ sequence for a wide range of star formation histories, and the results are no different for this two-infall model. 

Finally, in model (d) the metallicity of the infalling gas increases to ${\rm [X/H]}_{\rm CGM}=-0.7$ at late times. We choose this value because it is the maximum which can plausibly still reproduce the disk abundances; any higher and the infalling gas would be more metal-rich than the most metal-poor thin disk stars. This model results in very similar [O/Fe] distributions to the $y/Z_\odot=1$ case. We assume that the infalling gas has ${\rm [O/Fe]}=0$ at all times; an alternate run with ${\rm [O/Fe]}=+0.3$ shifted the distribution towards higher [O/Fe], worsening agreement with observations. In summary, pre-enriched gas infall may be necessary for the two-infall model to match the observed [O/Fe] distribution across the disk.

\begin{figure}
    \centering
    \includegraphics{figures/ofe_feh_density.pdf}
    \caption{The density of stars in the [O/Fe]--[Fe/H] plane from multi-zone models with (a) $y/Z_\odot=1$ and (b) $y/Z_\odot=2$, and (c) from the APOGEE DR17 catalog. Stars are restricted to the region defined by $7\leq R_{\rm gal}< 9$ kpc and $0.5\leq|z|<1$ kpc to highlight the $\alpha$-bimodality.}
    \label{fig:ofe-feh-density}
\end{figure}

Figure \ref{fig:ofe-feh-density} illustrates the origin of the intermediate-$\alpha$ peak predicted by the two-infall model at mid to high Galactic latitudes. Between $0.5\leq|z|<1$ kpc, both the models with $y/Z_\odot=1$ and $y/Z_\odot=2$ predict an over-density of stars near the abundance turn-over (${\rm [Fe/H]}\approx-0.3$, ${\rm [O/Fe]}\approx0.1-0.2$), which is not seen in the APOGEE sample. Additionally, it is clear that the shape of the low-$\alpha$ sequence in the model results (a concave-down ``comma'') is different from the data (a concave-up ``swoosh'').

\subsection{Abundance Evolution}
\label{sec:age-abundance}

\begin{figure*}
    \centering
    \includegraphics{figures/abundance_evolution.pdf}
    \caption{Stellar age--abundance relations produced by select multi-zone models. Each point represents a stellar population drawn from the Solar neighborhood near the midplane ($7\leq R_{\rm gal}\leq 9$ kpc, $0\leq |z| \leq 0.5$ kpc) and is color-coded by its birth radius. For visual clarity, we plot a only random mass-weighted sample of \num{10000} points in each panel. The black curve plots the ISM abundance at $R_{\rm gal}=8$ kpc over time. The red line segments plot the median abundance for APOGEE stars in 2 Gyr-wide age bins, and the shaded regions represent the 16th--84th percentiles in each bin. Age estimates from APOGEE stars come from \citet{leung_variational_2023}. Each column shows results from a different multi-zone model: {\bf (a)} our fiducial model, with $y/Z_\odot=1$, $\sigma_{\rm RM8}=2.7$ kpc, and pristine gas infall; {\bf (b)} a model with higher yields $y/Z_\odot=2$; {\bf (c)} a model with greater radial migration strength $\sigma_{\rm RM8}=5$ kpc; and {\bf (d)} a model which assumes the infalling gas has metallicity ${\rm [O/H]}_{\rm CGM}={\rm [Fe/H]}_{\rm CGM}=-0.7$.}
    \label{fig:abundance-evolution}
\end{figure*}

The dilution effect discussed in \ref{sec:yields} is clearly seen in the multi-zone model results. Figure \ref{fig:abundance-evolution} shows stellar age--abundance relations for four model outputs. Column (a) plots the output of our fiducial model, with $y/Z_\odot=1$, $\sigma_{\rm RM8}=2.7$ kpc, and $Z_{\rm CGM}=0$. The fiducial model shows two major discrepancies with the \citet{leung_variational_2023} age--abundance relation: a major dilution of $\sim0.5$ dex at a lookback time of $\sim9$ Gyr near where the data show a maximum in [O/H], and non-zero abundance evolution at late times where the data show very little abundance evolution. We also note the present-day metallicity is still sub-Solar in the fiducial model, but this can be corrected by decreasing $\eta$. A higher yield set (column b) mitigates both of these issues by shortening the time it takes the ISM metallicity to rebound post-$t_{\rm max}$, producing a much flatter abundance curve at late times. However, a consequence of these higher yields is a poor fit to the stellar [O/Fe] distribution, as discussed in Section \ref{sec:abundance-distributions}.

The observed rise in the median abundance of stars with ages of $\sim4-8$ Gyr is thought to be due to radial migration, as those stars are more likely to have migrated from the dense inner metal-rich regions of the Galaxy during that time. Although our fiducial model does include a prescription for radial migration, the majority of stars in that age range in column (a) have sub-Solar abundances. Column (c) of Figure \ref{fig:abundance-evolution} presents a model with $y/Z_\odot=1$ but with a stronger migration prescription of $\sigma_{\rm RM8}=5$ kpc. Here, the stars which make up the present-day Solar neighborhood are drawn from a wider range of birth $R_{\rm gal}$, producing a broader abundance distribution for any given age. However, even though this prescription is much stronger than the estimates of, e.g., \citet{frankel_measuring_2018}, the model cannot reproduce the observed age--abundance trends in [O/H] and [Fe/H].

Finally, as in Section \ref{sec:abundance-distributions}, we investigate a model where the gas infall is pre-enriched to ${\rm [O/H]}={\rm [Fe/H]}=-0.7$. As shown in column (d) of Figure \ref{fig:abundance-evolution}, this mitigates but does not completely solve the two discrepancies. The dilution effect of the second infall is reduced to the $\sim0.3$-dex level as the gas which replenishes the Galaxy's reservoir is no longer pristine; however, the width of the stellar abundance distribution is also reduced, \todo{because why?}. The late-time gas abundance evolution is similar to the fiducial model, but it ends at slightly super-Solar metallicity --- an effect which can be compensated by a slightly increased value of $\eta$. Overall, no modification to the $y/Z_\odot=1$ model is able to overcome both the dilution and late-time evolution issues.

\subsection{Abundances Across the Disk}
\label{sec:disk-abundances}

\section{Discussion}
\label{sec:discussion}

% \subsection{The SN Ia Delay-Time Distribution}
% \label{sec:dtd-discussion}

% \begin{figure*}
%     \centering
%     \includegraphics{figures/delay_time_distribution.pdf}
%     \caption{The effect of the SN Ia delay-time distribution (DTD) on chemical abundance tracks and stellar distributions. {\it Left:} The relative SN Ia rate as a function of time after star formation for each DTD model. The points illustrate the median delay times for each DTD. {\it Right:} Output of one-zone models which assume different SN Ia DTDs with the {\tt yZ2} yield set. The infall parameters are fixed at $\tau_1=1$ Gyr, $\tau_2=10$ Gyr, and $\tau_{\rm on}=3$ Gyr. {\bf Key takeaway:} the extent of the low-$\alpha$ sequence can be reduced by a more prompt SN Ia DTD, but the high-$\alpha$ sequence is also shifted to a more metal-poor regime.}
%     \label{fig:dtd}
% \end{figure*}

% Figure \ref{fig:dtd} illustrates the effect of the SN Ia DTD on the abundance predictions by the two-infall model. The fiducial plateau DTD produces the strongest high-$\alpha$ peak as well as the greatest separation between the two low-$\alpha$ peaks. The exponential DTD produces the most stars near Solar [O/Fe], but the low-$\alpha$ sequence is still distinctly bimodal. Finally, the power-law DTD shrinks the low-$\alpha$ loop such that there is only one low-$\alpha$ peak; however, this DTD produces high-$\alpha$ stars at metallicities much lower than observed, as discussed in \citet{dubay_galactic_2024}.

\subsection{Radial Gas Flows}
\label{sec:radial-flows}

\subsection{Star Formation Hiatus}
\label{sec:sfe-hiatus}

\begin{figure}
    \centering
    \includegraphics{figures/onezone_sfe_hiatus.pdf}
    \caption{Abundance tracks and distributions from one-zone models which experience an efficiency-driven starburst. The blue dashed curve represents the fiducial model that has an exponentially declining infall rate and constant star formation efficiency timescale $\tau_\star=2$ Gyr. The red solid curve plots the output of a model which experiences an enhancement of $\tau_\star$ by a factor of 10, for a duration of 200 Myr, starting at $t=1.4$ Gyr. Both models assume the {\tt yZ2} yield set, with $y_{\rm Fe}^{\rm Ia}$ reduced by 20\% to better match the model endpoint with the data, and $\eta=1.4$. The greyscale histogram presents the number density of APOGEE stars in the Solar annulus ($7\leq R_{\rm gal}\leq 9$ kpc, $0\leq|z|\leq2$ kpc) in [O/Fe]--[Fe/H] space, and the gray histograms in the marginal panels show the APOGEE stellar abundance distributions.}
    \label{fig:onezone-sfe-hiatus}
\end{figure}

The two-infall model falls into the broader category of GCE models which reproduce the $\alpha$-bimodality by halting or severely limiting star formation for some duration. For the two-infall model, this phase of low star formation immediately precedes the second infall epoch and is due to the relatively short timescale of the first infall epoch. However, as we have shown, the dilution of the ISM resulting from the second infall poses a challenge when comparing to age--abundance data.

A bursty infall history is not the only way to produce a gap in the star formation history. \citet{beane_rising_2024} present a simulated galaxy from the Illustris TNG50 suite that exhibits MW-like bimodality. They argue that the $\alpha$-bimodality is brought on by a brief ($\sim300$ Myr) quiescent period caused by bar formation. The virial mass of their galaxy grows steadily throughout this period, unlike in our two-infall model where the mass grows by a factor of \todo{X} during the 1 Gyr following the second infall.

While our semi-analytic model does not include a Galactic bar, we can explore the effects of a star formation hiatus by artificially boosting the SFE timescale $\tau_\star$ for a period of time. Figure \ref{fig:onezone-sfe-hiatus} illustrates the effect of this SFE-driven hiatus in a one-zone model with an exponentially declining infall rate. During the quiescent period, the [O/Fe] ratio slowly declines due to the delayed contribution of Fe from SNe Ia. Meanwhile, the gas mass continues to increase even as star formation is suppressed. When $\tau_\star$ is lowered at the end of the quiescent period, the high gas mass sparks a moderate star formation burst which causes stellar abundances to ``pile up'' at similar [O/Fe] values. The trough between the high- and low-$\alpha$ sequences results from the star formation returning to pre-quiescence behavior.

Our simple hiatus model offers a few parameters which control the chemical evolution. The onset time of the SFE hiatus controls the position of the high-$\alpha$ sequence: a later onset places the peak at lower [O/Fe]. The duration of the star formation hiatus \todo{(and the $\tau_\star$ enhancement factor?)} controls the strength of the high-$\alpha$ peak.

The parameters of the SFE hiatus in Figure \ref{fig:onezone-sfe-hiatus} were chosen to match the APOGEE stellar [O/Fe] distribution as closely as possible. However, there are some differences in detail, such as the dearth of stars at ${\rm [O/Fe]}\approx+0.35$ due to the star formation hiatus. We intend this model to illustrate another path to reproducing the $\alpha$-bimodality. Most of the high-$\alpha$ stars present in the Solar annulus have likely migrated from the inner Galaxy, where perhaps this SFE-driven hiatus was concentrated.

\section{Conclusions}
\label{sec:conclusions}

\section*{Acknowledgements}

\todo{Personal acknowledgements.}

LOD and JAJ acknowledge support from National Science Foundation grant no.\ AST-2307621. JAJ and JWJ acknowledge support from National Science Foundation grant no.\ AST-1909841.
LOD acknowledges financial support from an Ohio State University Fellowship.
JWJ acknowledges financial support from an Ohio State University Presidential Fellowship and a Carnegie Theoretical Astrophysics Center postdoctoral fellowship. \todo{Update grants and fellowships.}

Funding for the Sloan Digital Sky 
Survey IV has been provided by the 
Alfred P.\ Sloan Foundation, the U.S.\ 
Department of Energy Office of 
Science, and the Participating 
Institutions. 

SDSS-IV acknowledges support and 
resources from the Center for High 
Performance Computing  at the 
University of Utah. The SDSS 
website is \url{www.sdss4.org}.

SDSS-IV is managed by the 
Astrophysical Research Consortium 
for the Participating Institutions 
of the SDSS Collaboration including 
the Brazilian Participation Group, 
the Carnegie Institution for Science, 
Carnegie Mellon University, Center for 
Astrophysics | Harvard \& 
Smithsonian, the Chilean Participation 
Group, the French Participation Group, 
Instituto de Astrof\'isica de 
Canarias, The Johns Hopkins 
University, Kavli Institute for the 
Physics and Mathematics of the 
Universe (IPMU) / University of 
Tokyo, the Korean Participation Group, 
Lawrence Berkeley National Laboratory, 
Leibniz Institut f\"ur Astrophysik 
Potsdam (AIP),  Max-Planck-Institut 
f\"ur Astronomie (MPIA Heidelberg), 
Max-Planck-Institut f\"ur 
Astrophysik (MPA Garching), 
Max-Planck-Institut f\"ur 
Extraterrestrische Physik (MPE), 
National Astronomical Observatories of 
China, New Mexico State University, 
New York University, University of 
Notre Dame, Observat\'ario 
Nacional / MCTI, The Ohio State 
University, Pennsylvania State 
University, Shanghai 
Astronomical Observatory, United 
Kingdom Participation Group, 
Universidad Nacional Aut\'onoma 
de M\'exico, University of Arizona, 
University of Colorado Boulder, 
University of Oxford, University of 
Portsmouth, University of Utah, 
University of Virginia, University 
of Washington, University of 
Wisconsin, Vanderbilt University, 
and Yale University.

This work has made use of data from the European Space Agency (ESA) mission
{\it Gaia} (\url{https://www.cosmos.esa.int/gaia}), processed by the {\it Gaia}
Data Processing and Analysis Consortium (DPAC,
\url{https://www.cosmos.esa.int/web/gaia/dpac/consortium}). Funding for the DPAC
has been provided by national institutions, in particular the institutions
participating in the {\it Gaia} Multilateral Agreement.

% From the Center for Belonging and Social Change, https://cbsc.osu.edu/about-us/land-acknowledgement
We would like to acknowledge the land that The Ohio State University occupies is the ancestral and contemporary territory of the Shawnee, Potawatomi, Delaware, Miami, Peoria, Seneca, Wyandotte, Ojibwe and many other Indigenous peoples. Specifically, the university resides on land ceded in the 1795 Treaty of Greeneville and the forced removal of tribes through the Indian Removal Act of 1830. As a land grant institution, we want to honor the resiliency of these tribal nations and recognize the historical contexts that has and continues to affect the Indigenous peoples of this land.

\software{\vice \citep{johnson_impact_2020}, Astropy \citep{astropy_collaboration_astropy_2013,astropy_collaboration_astropy_2018,astropy_collaboration_astropy_2022}, scikit-learn \citep{pedregosa_scikit-learn_2011}, SciPy \citep{virtanen_scipy_2020}, Matplotlib \citep{hunter_matplotlib_2007}}

\appendix

\section{Reproducibility}
\label{app:reproducibility}

\todo{Blah.}

\bibliography{references}

\end{document}
